There are millions of applications to be found across app stores and the web, but the monolithic nature of the social media platforms means that our social interactions are mediated through a bare handful of dominant social applications that can't be modified or replaced.
In this work we explore the design and implementation of Graffiti, a \emph{pluralistic} infrastructure that aims to enable the emergence of a truly diverse ecosystem of inherently interoperable \emph{pure client} social applications that require no server implementation or significant resources beyond what is provided by the infrastructure.

%It minimizes barriers to creating, changing, and migrating between social applications.
Graffiti prioritizes user autonomy and design versatility while also addressing concerns like content moderation and privacy. Graffiti is designed with a specific eye towards preventing \emph{context collapse}, making it easy to create distinct spaces in which the different facets of a user's identity can be kept separate.

Graffiti is defined by its general-purpose API and we have realized it with a decentralized backend, frontend tooling, and many example applications.
Graffiti includes a general-purpose backend and a frontend library that together allow for diverse, powerful, and interoperable social applications to be built with only HTML, CSS and Javascript.  We provide both centralized and decentralized implementations of Graffiti, with discussion of the scalability, economics, and potential for abuse of each.
We evaluate Graffiti by examining a case-study ecosystem of example applications.
