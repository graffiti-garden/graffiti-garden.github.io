
% the closed APIs, network effects of modern social applications
% have forced most people onto a small number of applications
% over which they have no control.
% The platform-centric design,
% closed APIs, and network effects
% of modern social applications force
% most people onto a small number of dominant
% social platforms over which they have next to no control.

% Add more "color" to the problem.

% There are already problems

People today have little autonomy over the social applications
they inhabit.
We present the Graffiti API, which can be used
to build a wide variety of custom social applications with
relative ease and pure client-side development. These applications
also \emph{interoperate}, allowing people to freely move between
a plurality of designs---each with its own aesthetic, feature set,
and moderation---all without ``losing their friends.''

Our concept of \emph{total reification} makes it possible
for seemingly contradictory designs, including conflicting
moderation rules, to interoperate.
Conversely, our concept of \emph{channels}
prevents interoperation from occurring by accident, avoiding context collapse.

The Graffiti API is not tied to any specific communication protocol.
We present two decentralized protocols for it,
but others could be developed and used alongside them,
much like how HTTP and HTTPS coexist on the web.
Above the API, we built a Vue.js plugin, which we used
to develop applications similar to Twitter, Signal, and Wikipedia.
Our case studies explore how these and other novel applications interoperate,
as well as the ecosystems that Graffiti enables.
