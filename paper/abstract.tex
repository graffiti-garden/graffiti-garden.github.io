% Most social applications today offer limited personalization options
% and the workaround of creating a new social application is
% not only technically challenging but also results in an application
% that is siloed away from existing communities.
% We present \emph{Graffiti}, which can be used to build a
% wide variety of personalized social applications
% with relative ease that also interoperate with each
% other. People can freely move between a plurality of
% designs---each with its own aesthetic, feature set, and moderation---all
% without losing their friends or data.

% the closed APIs, network effects of modern social applications
% have forced most people onto a small number of applications
% over which they have no control.
% The platform-centric design,
% closed APIs, and network effects
% of modern social applications force
% most people onto a small number of dominant
% social platforms over which they have next to no control.

% Add more "color" to the problem.

% There are already problems

%DK The challenge of constructing entirely new social applications and convincing users to migrate to them leaves us stuck with a poverty of rigid, one-size-fits-all applications.  

% The difficulty of making social applications and the social pressure to ``lock-in'' to a particular application makes converge makes most social applications highly impersonal one-size-fits-all designs.
{\sethlcolor{pink}\hl{%
Most social applications, from Twitter to Wikipedia,
have rigid one-size-fits-all designs, but building new social applications
is both technically challenging and results in
applications that are siloed away from existing communities.
}}%
We present \emph{Graffiti}, which can be used
to build a wide variety of personalized social applications
with relative ease that also interoperate with each other. People can freely move between
a plurality of designs---each with its own aesthetic, feature set,
and moderation---all without losing their friends or data.

Our concept of \emph{total reification} makes it possible
for seemingly contradictory designs, including conflicting
moderation rules, to interoperate.
Conversely, our concept of \emph{channels}
prevents interoperation from occurring by accident, avoiding context collapse.

% The Graffiti API il not tied to any specific communication protocol. We present two viable decentralized protocols for it, but others could be developed and used alongside them, much like how HTTP and HTTPS coexist on the web. Above the API, we built a Vue.js plugin, which we used to develop applications similar to Twitter, Signal, and Wikipedia. Our case studies explore how these and other novel applications interoperate, as well as the ecosystems Graffiti enables.

{\sethlcolor{pink}\hl{%
Graffiti applications interact through a minimal \emph{client-side API}
which we show admits at least two \emph{decentralized} implementations.
}}%
Above the API, we built a Vue.js plugin, which we use to
develop applications similar to Twitter, Messenger, and Wikipedia
using only client-side code.
Our case studies explore how these and other novel applications interoperate,
as well as the broader ecosystem that Graffiti enables.
