% Add more "color" to the problem - Josh

Most social applications, from Twitter to Wikipedia,
have rigid one-size-fits-all designs, but building new social applications
is both technically challenging and results in
applications that are siloed away from existing communities.
We present \emph{Graffiti}, a system that can be used
to build a wide variety of personalized social applications
with relative ease that also interoperate with each other. People can freely move between
a plurality of designs---each with its own aesthetic, feature set,
and moderation---all without losing their friends or data.

Our concept of \emph{total reification} makes it possible
for seemingly contradictory designs, including conflicting
moderation rules, to interoperate.
Conversely, our concept of \emph{channels}
prevents interoperation from occurring by accident, avoiding context collapse.

Graffiti applications interact through a minimal \emph{client-side API},
which we show admits at least two \emph{decentralized} implementations.
Above the API, we built a Vue plugin, which we use to
develop applications similar to Twitter, Messenger, and Wikipedia
using only client-side code.
Our case studies explore how these and other novel applications interoperate,
as well as the broader ecosystem that Graffiti enables.
