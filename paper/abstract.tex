
% the closed APIs, network effects of modern social applications
% have forced most people onto a small number of applications
% over which they have no control.
% The platform-centric design,
% closed APIs, and network effects
% of modern social applications force
% most people onto a small number of dominant
% social platforms over which they have next to no control.

% Add more "color" to the problem.

% There are already problems

% The difficulty of making social applications and the social pressure to ``lock-in'' to a particular application makes converge makes most social applications highly impersonal one-size-fits-all designs.
People today have little autonomy over the social applications
they inhabit.
We present \emph{Graffiti}, which can be used
to build a wide variety of personalized social applications
with relative ease and pure client-side development. These applications all \emph{interoperate}, allowing people to freely move between
a plurality of designs---each with its own aesthetic, feature set,
and moderation---all without ``losing their friends'' or their data.

Our concept of \emph{total reification} makes it possible
for seemingly contradictory designs, including conflicting
moderation rules, to interoperate.
Conversely, our concept of \emph{channels}
prevents interoperation from occurring by accident, avoiding context collapse.

% The Graffiti API il not tied to any specific communication protocol. We present two viable decentralized protocols for it, but others could be developed and used alongside them, much like how HTTP and HTTPS coexist on the web. Above the API, we built a Vue.js plugin, which we used to develop applications similar to Twitter, Signal, and Wikipedia. Our case studies explore how these and other novel applications interoperate, as well as the ecosystems Graffiti enables.

Graffiti applications interact through an \emph{API} that can be implemented over a variety of different \emph{protocols} (we present two) that can all be coexist, much as HTTP and HTTPS coexist on the web.
% We present two decentralized protocols which we have implemented, including one that uses only commodity storage platforms such as dropbox to store and distribute content.
Above the API, we built a Vue.js plugin, which we used
to develop applications similar to Twitter, Messenger, and Wikipedia.
Our case studies explore how these and other novel applications interoperate,
as well as the ecosystems that Graffiti enables.
