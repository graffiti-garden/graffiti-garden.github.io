\section{API}
\label{api}

We now describe our API that fulfills the requirements
outlined in Section~\ref{requirements} and builds upon the concepts
just discussed in Section~\ref{concepts}.
The API features simple methods for \texttt{login} and \texttt{logout}
and a series of CRUD\footnote{Create, Read, Update, Delete} methods for manipulating objects:
\texttt{put}, \texttt{get}, \texttt{patch}, and \texttt{delete}.
The novel \texttt{discover} method allows users to query for objects
from a set a of channels.
Finally, there are two specialized ``recovery'' methods included for completeness:
\texttt{recoverOrphans} and \texttt{channelStats},
which prevent users from ``losing'' objects or channels, respectively.
Overall, as much complexity as possible is pushed \emph{below} the API,
while as much functionality as possible is pushed \emph{above}.

The API is written as an abstract TypeScript class, allowing
developers building on top of the API to swap out different implementations
with a single line of code.
The API's complete documentation, specification, and test suite is available online\footnote{
    \url{https://api.graffiti.garden/classes/Graffiti.html}
}, and so here we focus on the important design decisions.

TypeScript is a type annotation layer around standard JavaScript
and provides enormous benefit to developers working on large projects.
However, the Graffiti API is still completely functional
in a regular JavaScript environment and entire Graffiti applications can be written
with vanilla HTML, CSS, and JavaScript that function in the browser without
any build step.
These web technologies can also be used to build desktop applications
with Electron\footnote{
    \url{https://www.electronjs.org/}
} or mobile applications as progressive web applications\footnote{
    \url{https://developer.mozilla.org/en-US/docs/Web/Progressive_web_apps}
}.

\subsection{Logging In}

While it is possible to retrieve public content without logging in,
all other functionality, including
creating and modifying content or reading access-controlled content,
requires proof that the user owns a particular \emph{actor}, thus requiring a login.

Modern log in methods, like OAuth 2.0, while secure, are incredibly complex~\cite{oauth}.
To hide this complexity from application developers and to make the API future proof
to evolving security standards, the API provides a simple \texttt{login} method that works as follows:

\begin{enumerate}
\item
A user performs an action, like clicking a ``log in'' button, that triggers
the application to call the \texttt{login()} method.
\item
In response, the implementation of the API may do any manner of things,
including opening up a pop up or redirecting the user to a new page.
In that \allowbreak{implementation-specific} interface, which the application developer
does not need to build, the user might be asked to provide credentials like a username
and password or a public key pair.
See Figure~\ref{above-and-below:figure:login} for our implementation.
\item
Once the user is authenticated they are redirected back to the application, if applicable,
and a JavaScript \emph{event} fires that includes
a \texttt{session} object. The \texttt{session} object includes the authenticated \texttt{actor}
URI and other information that the implementation needs to verify authentication, like
a token or a signing function, however those details are entirely opaque to the application developer.
\item
When the application wants to perform an action that requires authentication,
it must provide the \texttt{session} object. For example,
\texttt{delete(object, session)}.
\item
Finally, to log out, a user simply needs to call \texttt{logout} on the appropriate
\texttt{session} object: \texttt{logout(session)}.
\end{enumerate}

We chose to keep the \texttt{session} object external from the \texttt{Graffiti} class
to prevent developer mode errors, where an application attempts to perform an action
that requires authentication without asking the user to log in first.
These mode errors are detected by static type checking.
Separating the \texttt{session} object also allows an application to be logged in to
multiple actors at once, which can be useful for managing pseudonyms or anonymous interactions.

\subsection{Objects and CRUD}

Objects, as overviewed in Section~\ref{concepts:objects}, include
the following meta properties:
\begin{enumerate}
\item
\texttt{value} (object): The actual data, which can be any JSON object~\cite{json}.
\item
\texttt{url} (string): A string to uniquely identify and locate the object.
\item
\texttt{actor} (string): The URI of the actor that created the object.
\item
\texttt{channels} (array of strings): A list of channels that the object is contained in.
\item
\texttt{allowed} (array of strings \hl{or undefined}): An \hl{optional} list of actor URIs that are allowed to access the object.
\hl{When undefined, any actor can access the object if they know either its \texttt{url} or one of its \texttt{channels}.}
\item
\texttt{revision} (number): A Unix time\-stamp, Lamport time\-stamp~{\cite{lamport}}, or other counter that
increases with each modification of the object.
\end{enumerate}

Subtly, \texttt{channels} and \texttt{allowed} do not necessarily
reflect \emph{all} the channels an object is contained in, or
\emph{all} the actors that are allowed access the object.
Like a BCC email, it can be useful to send a message to multiple parties
without revealing their existence to one another, preventing
context collapse.
So \texttt{channels} and \texttt{allowed} are both \emph{masked} to any
actor other than the object's owner so that they only show information an actor implicitly knows,
like the channel their application queried to discover the object.
However an application can always
add metadata, including channels or allowed lists, to the object's \texttt{value}
to reveal additional information.

Objects can be interacted with through familiar CRUD methods:
\texttt{put}, \texttt{patch}, and \texttt{delete} for modifying objects,
and \texttt{get} for retrieving objects.
The modification methods each require a \texttt{session} whereas
for \texttt{get} the \texttt{session} is optional,
enabling a user to view content without being logged in.
The API responds identically for an object that does not exist,
one that has been deleted, or one that the actor is not allowed to see,
so that no information about an object's status is leaked.

The \texttt{url} and \texttt{actor} are assigned to an object when it is first \texttt{put}
and are immutable.
As mentioned in our discussion of total reification in Section~\ref{concepts:total-reification},
only the actor that created an object can mutate the object at that \texttt{url}.
An object's owner can arbitrarily mutate the object's
\texttt{value}, \texttt{channels}, and \texttt{allowed} list.
The \texttt{revision} field is increased by the underlying implementation on each modification.

We expect that the ecosystem will contain many posts that a particular application cannot understand and will wish to ignore.  To simplify this process,
\texttt{get} requires the application developer to specify
a JSON schema, a widely used and flexible standard for describing
the structure of JSON objects~\cite{jsonschema}.
Any object that does not match the provided schema is rejected.
The API includes type inference on these schemas which makes it possible to work
with objects as if they were strongly typed, even though they can take on arbitrary values.

\begin{minted}{javascript}
const object = await graffiti.get(url, {
  value: {
    required: ["content"],
    properties: {
      content: { type: "string" },
    }
  }
}); // Error if the object does not match

const content = object.value.content; // string
\end{minted}

By default, JSON schemas also match objects that contain additional,
unspecified properties, allowing developers to freely
add new object properties without breaking
interoperability with existing applications,
encouraging the folksonomy.

\subsection{Discover}

\texttt{discover} lets an application developer query for objects
from a particular set of channels.
Objects are returned from \texttt{discover} as an \emph{asynchronous generator}.
This allows the application to process objects as they come in,
using a \texttt{for\ldots{await}} loop, rather than waiting for all objects
to be fetched. The generator also hides the potential complexity of
query pagination below the API.

Like \texttt{get}, \texttt{discover} can be given an optional \texttt{session}
to access private objects, and it requires a JSON schema.
The schema,
in addition to providing type safety, makes \texttt{discover} far more efficient
as a query mechanism, limiting the objects returned to only those that interest the particular application.
For example, the channel named by an actor's URI has many overloaded uses,
including being the site for the actor to broadcast Twitter-like posts,
and the site where they receive direct messages.
To receive only direct messages from Alice,
sent over the past day, a developer may do the following:

\begin{minted}{javascript}
const iterator = graffiti.discover(
  [session.actor], // Look in "my" channel
  {
    value: {
      required: ["content", "published", "to"],
      properties: {
        content: { type: "string" },
        published: {
          type: "number",
          minimum: Date.now() - 24 * 60 * 60 * 1000
        }
        to: { const: session.actor }
      }
    },
    actor: { const: "https://alice.example.com" },
  },
  session
)

for await (const object of iterator) {
  console.log("alice → me:", object.value.content);
}
\end{minted}

Reified activities can also be processed to construct more
complex \texttt{discover} schemas.
For example, the schema may exclude objects by
actors who have been ``blocked'' by a sufficient
number of other actors.
%DK really?  this seems like for more than a schema language, turning into a general graph-database query language?

The \texttt{discover} iterator will end once all objects
currently matching the channels and schema have been fetched.
It ends by returning a
\texttt{continue()} function,
which can be used to resume
the \texttt{discover} call.\footnote{
    A \texttt{cursor} string is also returned, which can be
    serialized for when an application closes and reopens.
    The \texttt{cursor} can then be passed into a \texttt{continueObjectStream}
    method and will produce the same result as \texttt{continue()},
    however it is does not preserve type safety and is less convenient.
}

\texttt{continue()} returns new objects, changed objects, \emph{and}
tombstones for objects that have been deleted or modified so that they no longer match the
specified channels, schema, or actor.
In total, these allow an application to keep an up to date state
without having to re-fetch all objects.
A tombstone only includes an object's URL and the time of deletion.
Tombstones must inevitably expire to preserve a user's right to be forgotten,
causing \texttt{continue()} to throw an error
that signals to the application to poll \texttt{discover} from scratch.

\texttt{continue()} can be recursively called for applications,
like direct messaging, that require real-time updates.
Below the API, an implementation might perform complex rate limiting on these recursive calls or
adaptively switch to a more efficient push-based transport, like a WebSocket, but this is hidden from developers working above.

\subsection{Recovery Methods}

It is intentionally impossible to get a global view of all objects
in Graffiti, to prevent context collapse. Most of the time,
an application must know either an object's URL or one of the channels
the object is contained in to retrieve it. However, we relax this
constraint for an actor's \emph{own} objects, to prevent them from
``losing'' objects.

The \texttt{recoverOrphans} method finds objects that an actor has created
that are not contained in \emph{any} channel, and \texttt{channelStats}
returns a list of all the channels that an actor has posted in.
Both methods output streams similar to \texttt{discover}
and require a \texttt{session}.
%DK ATP, how about instead allowing a null argument to discover which means that you are not looking in any channel but are instead looking through all your own posts (possibly filtering by a schema specification, which could for example indiciate that you only want posts not in any channel)

These methods are not intended to be used in most applications but are
necessary for creating tools like Facebook's ``activity log,''
\hl{or other novel digital hygiene tools},
that allow users to audit their system-wide usage.
