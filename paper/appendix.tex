\section{More Channel Name Constructions}

Graffiti channels, as described in
Section~\ref{concepts:channels},
are designed to allow for new contextual meanings to evolve according to a folksonomy.
% with new channel name constructions.
In Table~\ref{concepts:channels-and-applications},
we listed some of those possible meanings.
Here, we expand on how continuous
contexts, like locations, can be represented with
discrete channel names and how channels names
can be combined to create new
intersectional meanings.
% we expand on how continuous contexts like
% locations might be represented in channels and how
% channels could be combined to make new channels.

% can evolve as a folksonomy to include
% different meanings and different name constructions.
% We provide a few more examples of channel name constructions
% for contexts that are not obvious.

There are many standard models for representing
discrete physical \emph{areas} that could be co-opted to construct channel
names for location-centric contexts. These include
human-generated divisions, like zip codes,
and automatic divisions like what3words\footnote{
    \url{https://what3words.com}
}, a tiling of the earth into 3 meter squares.
The construction chosen for a particular application will
depend on that application's purpose,
its ``reach'' needs, and its privacy needs.
For example, a civic discussion application
might benefit from using voting precincts and wards
to construct the channel names.
A classifieds application, like Craigslist, may benefit from posting
advertisement objects to channels representing multiple \emph{scales}
in a ``resolution pyramid,'' to encourage discoverability,
using a hierarchical area standard such as H3\footnote{
    \url{https://h3geo.org/}
}.
A location-based dating application, like Tinder, may intentionally
post profile objects only to coarse area channels to prevent stalking.
Conversely, when exact location is important,
as is the case for geocaching, an application could post the geocache
object to several coarse area-based channels to allow users
to search for geocaches nearby, but include
the exact cache location as an object \emph{property}
so that receiving applications can mark the cache on a map.

Channels can also be concatenated to produce new, more specific channels,
which may help to prevent context collapse in overloaded domains.
For example, a Tinder-like application may post profile objects
to channels with names of the form
\texttt{dating+\allowbreak{}zip:\allowbreak{}12345} rather than
\texttt{zip:\allowbreak{}12345} so that dating profiles
do not accidentally show up to neighbors browsing a local
discussion platform, like Nextdoor.
% Of course, the neighbors may
Concatenation may also happen in natural language.
For example, on Reddit,
programming-related memes are posted to \texttt{r/\allowbreak{}Programmer\allowbreak{}Humor}
rather than \texttt{r/\allowbreak{}programming+\allowbreak{}r/funny}.

% Posting to multiple channels, as in our example of posting
% to multiple areas scales, effectively allows posting
% to the \emph{union} of contexts while concatenation allows users
% to post to the \emph{intersection} of contexts.
% Prior to settling on the channels concept we
% considered an alternate construction
% that allows arbitrary logical formulas to be represented.
% In this construction, rather than providing an array of channels,
% the developer would supply one or more \emph{near misses}.

% This construction allows for the creation of arbitrary \emph{logical}
% intersections but the real-world use cases of those are limited.

\section{Announce Protocol}

\label{above-and-below:announce-protocol}

The remote protocol, described in Section~\ref{above-and-below:remote-protocol},
is currently built with a \emph{registry} and applications must
call \texttt{discover} on all the servers in the registry for
each high-level \texttt{discover} call.
This has the potential to be inefficient if the application queries
servers that have no objects, and also adds friction to participation,
as each new server needs to be ``registered'' to become discoverable
to the rest of the network.

Alternatively, it is possible to use an ``announce'' protocol,
which performs a similar service to that of the ``tracker'' used in
the commodity storage protocol (see Section~\ref{above-and-below:commodity-storage-protocol}). However, the announce protocol is built on top of the isolated Graffiti
API that is exposed by each remote server, without the need for a separate
tracker service. Building on top of the Graffiti API also 
allows the announce protocol to flexibly evolve as
new object properties are introduced.

The announce protocol assumes that the network is distributed like other
federated services like Email or Mastodon, with a small handful of
massive ``well-known'' servers, like GMail or mastodon.social, and a long
tail of smaller, possibly self-hosted servers~\cite{mastodonchallenges}.
When an actor publishes an object to one such small server,
their client also publishes an \texttt{"announce-server"}
activity, pointing to their small server, on several of the well-known
servers:
\begin{minted}{javascript}
{
  activity: "announce-server",
  target: "https://my-small-server.com"
}
\end{minted}
This announcement must be crossposted to all the channels
that the actor posts to on the small server.
If there are channels where the actor has only posted
access-controlled objects, they must post additional announcement
objects with appropriate \texttt{allowed} lists.

To fulfill a high-level \texttt{discover} request,
the client must first \texttt{discover} for 
\texttt{"announce-server"} activities in the channels of
the high-level request
on each of the well-known servers.
Then, the client can \texttt{discover} for objects
matching the high-level channels \emph{and} schema on each of the targeted servers.
The protocol will not miss objects, so long as their
is an overlap between the retrieving and posting actors' well-known
servers lists.
The protocol will also not make requests to any server
that does not have objects in the channels of interest.
The protocol could be updated to also include object
\emph{schemas} in the announcements, as an additional
filter to prevent querying unnecessary servers.
This \texttt{"schemas"} property and other properties
can be flexibly added to \texttt{"announce-server"} activities
in a folksonomic way, like any other Graffiti objects.

We developed a proof-of-concept implementation of the announce protocol\footnote{
    See supplementary materials.
},
however, it is not currently in use as the Graffiti ecosystem is not
large enough for it to be relevant.
It is not possible for the commodity-storage protocol to implement
the announce protocol, removing the need for a tracker,
because commodity storage servers do not allow clients to
query multiple actors' data at once.

One downside of the announce protocol is that it leaks all
channel names to the well-known servers.
This may not be desirable to an individual or community that
decides to self-host specifically for privacy purposes.
This can be resolved by posting announcements to the channels
named by the \emph{hashes} of the actual channel names.
Alternatively, if the remote servers become end-to-end-encrypted,
as mentioned in Section~\ref{above-and-below:below:future-work},
this would no longer be an issue.

% By implementing the tracker using Graffiti objects,
% it is possible to flexibly add new features to the tracker
% in the same way that it is possible to flexibly add
% functionality to any other Graffiti object.

% \subsubsection{Delegation and Portability}
% \label{above-and-below:delegation}

% When a remote server returns an object that it claims
% was created by a particular actor, how can the client verify that claim?
% Many decentralized systems use public key cryptography to sign messages,
% however this has the potential to disable \emph{repudiation},
% violating requirement~\ref{requirements:forgiving}.

% Instead, we use delegation. When a user first \texttt{put}s an object
% to a new remote server, they mark that server as ``delegated'' in their
% webId \emph{document}. This document is public and so others can verify
% that the user has delegated permission to the server.
% Once an actor deletes an object from a server there is no
% ``paper trail'' still connecting them to the object.

% The webId document can also be used to set up "forwarding rules" for
% situations where an actor wants to move their data from one server to
% another without breaking all their object's \texttt{url}s.
% This enables \emph{data portability}.

% While our use of objects is similar to ActivityPub, one subtle difference is
%     that the objects are ``ducktyped'' unlike ActivityPub's use of \texttt{type} properties.
%     This allows for slightly more flexibility and interoperability.
% },

% \subsubsection{Notes}


% \subsubsection{Near Misses}

% Compared to near misses, the channel concept allows data recipients to only make queries on the \emph{union} of attributes --- a query can be for posts by Alice \emph{or} in reply to the target post but not for posts by Alice \emph{and} in reply to the target post. The recipient is left to do further filtering and in our implementation section we discuss the consequences this has on performance and client-side only design.

% It is possible to \emph{intersect} channels by concatenating them.
% For example, if we only want a message to be seen by people interested in dating at a particular
% location the channel would be
% The order of the concatentation is ambiguous and there is the potential.

% These union queries also limit the options that a poster has to specify where they want data to appear. However, we believe that in most cases where a user wants content to only be available in the intersection of two channels, it is appropriate to create a new channel through either automated or natural language concatenation.

% One possible generalization of `Channels' is near misses.
% The concept is hard to develop and the scope in which such logical interections
% might be useful is limited. However for completness we thought it is worth mentioning.

% BCCs don't work with near misses.

% One way to operationalize public contexts is similar to how people set boundaries in natural language.
% When asked a general question, like ``How are you?'', many people respond with a general answer, ``Good, and you?''
% When asked a more specific question, like ``What did you think of the game last night?'', many people respond with a more specific answer.

% Asking a more specific question can be a shibolleth
% signals the querier's interests
% which can give the responder confidence they are not oversharing. A more specific question may also signal whether the querier is part of some in-group, allowing the responder to disclose information that may technically be public but not readily shared (``Are you a friend of Dorothy's?'').

% This natural language analogy is a far cry from how database queries work.
% For example, in the Nostr infrastructure that allows users to make general queries on each others social data, a simple query for a users ID returns everything they've ever posted. This is the equivalent of asking someone ``How are you?'' and then sitting through their entire life history.

% What we are suggesting is that somehow when a user posts data, they should also be able to outline \emph{how specific} a query must be in order to retrieve that data.
% This way, while the data is still public, it won't be found by users not asking the ``right question.''
% Again, this is not about absolute privacy, which can be handled via standard access control, but about giving a content creator situational privacy about the audience that will consume their data.

% A first attempt would be for users to mask out fields they don't want queried.
% For example, suppose a user \texttt{alice@example.com} wants to reply to a post that can be identified with the URL \texttt{example.com/posts/12345}.
% Using the Activity Vocabulary, we can represent Alice's reply as follows:

% \begin{verbatim}
% {
%    type: 'Note',
%    content: "Thank you for a great post!",
%    inReplyTo: example.com/posts/12345,
%    actor: alice@example.com
% }
% \end{verbatim}

% Alice masks out all fields except for \texttt{inReplyTo} and gives the redacted post to an indexer.
% When someone queries for \texttt{\{ actor: alice@example.com \}} they won't be able to find the data, but when they query for \texttt{\{ inReplyTo: example.com/posts/12345 \}} they will.
% Unfortunately, if a friend of Alice's is only interested in her replies and queries for the appropriate actor \emph{and} reply fields, they won't see Alice's reply.

% One way to solve this generally and consistently by borrowing the concept of \emph{near misses} from classical artificial intelligence.
% When training a classifier, a near miss is something that is very close to being an example of a class of the concept without \emph{actually} being an example of that class.
% We can apply this to queries: to retrieve a piece of data a query must match the data, but must \emph{not} match the near misses.

% In our example, Alice's near miss should be examply the same as her post data, only
% the \texttt{inReplyTo} field should be changed to a random URL.
% A query for Alice's actor will match both will match both the reply \emph{and} its near miss, and so the reply \emph{will not} be returned.
% A query for posts \texttt{inReplyTo} the target post will match the reply but \emph{not} its near miss and so the reply \emph{will} be returned.
% Importantly, queries for posts that are both by Alice \emph{and} replies to the target post will also match the reply, but not its near miss and so will be returned.

% This system is quite general and with multiple near misses it can be used to qualify ``how specific'' a query is with arbitrary logical intersections.
% However, we found in initial prototypes that its contrapositive formulation is extremely confusing.
% Moreover, aside from some contrived examples, all use cases could be described by a simpler concept: \emph{channels}.


% Websites make it possible for everyone to annotate the web.
% In fact, Graffiti's namesake comes from an article
% describing an early web annotation system, Third Voice,
% as Graffiti on the web.
% The authors meant it as a slur, but we believe graffiti
% can be a beautiful art form and a powerful tool for social change.


% One particular benefit of the channel concept over near misses, is that channels more naturally allow for BCCing.
% While many of the channels a user posts to are presumably related to properties of the data itself, it is also possible to post data to multiple channels without revealing those channels to each other.
% For example, a user may be part of several different contexts invite them to a party with a universal thread, but doesn't want to inundate one with another by including them directly.


