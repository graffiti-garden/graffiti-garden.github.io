
While our use of objects is similar to ActivityPub, one subtle difference is
    that the objects are ``ducktyped'' unlike ActivityPub's use of \texttt{type} properties.
    This allows for slightly more flexibility and interoperability.
},

\subsubsection{Notes}

One particular benefit of the channel concept over near misses, is that channels more naturally allow for BCCing.
While many of the channels a user posts to are presumably related to properties of the data itself, it is also possible to post data to multiple channels without revealing those channels to each other.
For example, a user may be part of several different contexts invite them to a party with a universal thread, but doesn't want to inundate one with another by including them directly.


\subsubsection{Near Misses}

Compared to near misses, the channel concept allows data recipients to only make queries on the \emph{union} of attributes --- a query can be for posts by Alice \emph{or} in reply to the target post but not for posts by Alice \emph{and} in reply to the target post. The recipient is left to do further filtering and in our implementation section we discuss the consequences this has on performance and client-side only design.

It is possible to \emph{intersect} channels by concatenating them.
For example, if we only want a message to be seen by people interested in dating at a particular
location the channel would be
The order of the concatentation is ambiguous and there is the potential.

These union queries also limit the options that a poster has to specify where they want data to appear. However, we believe that in most cases where a user wants content to only be available in the intersection of two channels, it is appropriate to create a new channel through either automated or natural language concatenation.

One possible generalization of `Channels' is near misses.
The concept is hard to develop and the scope in which such logical interections
might be useful is limited. However for completness we thought it is worth mentioning.

BCCs don't work with near misses.

One way to operationalize public contexts is similar to how people set boundaries in natural language.
When asked a general question, like ``How are you?'', many people respond with a general answer, ``Good, and you?''
When asked a more specific question, like ``What did you think of the game last night?'', many people respond with a more specific answer.

Asking a more specific question can be a shibolleth
signals the querier's interests
which can give the responder confidence they are not oversharing. A more specific question may also signal whether the querier is part of some in-group, allowing the responder to disclose information that may technically be public but not readily shared (``Are you a friend of Dorothy's?'').

This natural language analogy is a far cry from how database queries work.
For example, in the Nostr infrastructure that allows users to make general queries on each others social data, a simple query for a users ID returns everything they've ever posted. This is the equivalent of asking someone ``How are you?'' and then sitting through their entire life history.

What we are suggesting is that somehow when a user posts data, they should also be able to outline \emph{how specific} a query must be in order to retrieve that data.
This way, while the data is still public, it won't be found by users not asking the ``right question.''
Again, this is not about absolute privacy, which can be handled via standard access control, but about giving a content creator situational privacy about the audience that will consume their data.

A first attempt would be for users to mask out fields they don't want queried.
For example, suppose a user \texttt{alice@example.com} wants to reply to a post that can be identified with the URL \texttt{example.com/posts/12345}.
Using the Activity Vocabulary, we can represent Alice's reply as follows:

\begin{verbatim}
{
   type: 'Note',
   content: "Thank you for a great post!",
   inReplyTo: example.com/posts/12345,
   actor: alice@example.com
}
\end{verbatim}

Alice masks out all fields except for \texttt{inReplyTo} and gives the redacted post to an indexer.
When someone queries for \texttt{\{ actor: alice@example.com \}} they won't be able to find the data, but when they query for \texttt{\{ inReplyTo: example.com/posts/12345 \}} they will.
Unfortunately, if a friend of Alice's is only interested in her replies and queries for the appropriate actor \emph{and} reply fields, they won't see Alice's reply.

One way to solve this generally and consistently by borrowing the concept of \emph{near misses} from classical artificial intelligence.
When training a classifier, a near miss is something that is very close to being an example of a class of the concept without \emph{actually} being an example of that class.
We can apply this to queries: to retrieve a piece of data a query must match the data, but must \emph{not} match the near misses.

In our example, Alice's near miss should be examply the same as her post data, only
the \texttt{inReplyTo} field should be changed to a random URL.
A query for Alice's actor will match both will match both the reply \emph{and} its near miss, and so the reply \emph{will not} be returned.
A query for posts \texttt{inReplyTo} the target post will match the reply but \emph{not} its near miss and so the reply \emph{will} be returned.
Importantly, queries for posts that are both by Alice \emph{and} replies to the target post will also match the reply, but not its near miss and so will be returned.

This system is quite general and with multiple near misses it can be used to qualify ``how specific'' a query is with arbitrary logical intersections.
However, we found in initial prototypes that its contrapositive formulation is extremely confusing.
Moreover, aside from some contrived examples, all use cases could be described by a simpler concept: \emph{channels}.


Websites make it possible for everyone to annotate the web.
In fact, Graffiti's namesake comes from an article
describing an early web annotation system, Third Voice,
as Graffiti on the web.
The authors meant it as a slur, but we believe graffiti
can be a beautiful art form and a powerful tool for social change.
