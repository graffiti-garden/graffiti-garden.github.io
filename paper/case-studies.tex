\section{Case Studies}

% All are built with our Vue plugin and
% pure client-side applications with no servers
% involved other than those interfaced with below the Graffiti API.

% The first demonstrate interoperability.
% The latter demonstrate how ``weird'' social applications
% can become when there is not central moderator.
\begin{enumerate}
\item
Wide variety/diversity
\item
Interoperability and interplay
\item
``Weird'' with no central moderator
\end{enumerate}
All of the applications were written with pure client-side code on
top of the Graffiti Vue plugin.

\subsection{Namebook and Gloof}

Namebook is a text-centric microblogging platform,
where users can post updates, follow others,
etc.
It also includes a directory of users, that users
can choose to put themselves on, but not necessary.
Threaded replies

Gloof is an application for sharing flyers to events in the [REDACTED]-area.
The flyers are displayed in a grid, similar to Instagram,
but there is just one feed, rather than.

but the flyers are published from an explicit set of approved members.
Gloof has a built-in set of administrators.
Anyone can comment on Gloof posts but the Gloof
moderators can selectively delete posts.

Namebook posts under a user's channel as described in section~\ref{concepts:channels}.
Gloof posts are posted to the channel \texttt{"gloof"} and filtered
for only their approved.
Gloof posts can be optionally cross-posted to Namebook.
Namebook cannot parse the images but can still read the
text description.
Additionally, comments on Namebook are Twitter-like
and available in a replies tab, unlike Gloof comments
which only show up under the post

Additionally, Gloof posts are posted with a timestamp.
This allows those posts to interoperate with a
separate calendar application.
Similar applications might also post the data in a map,
however this app explicitly hides locations.

% Posts to Gloof can be cross-posted to Namebook.

% Users of [Redacted] can are "cross-posted" to Namebook.
% Comments under than post show up on
% both Namebook and [Redacted]. However,
% Comments on Namebook \emph{also} show up
% in the users profile. THis is not the case
% for comments on Redacted which, like INstagra

% Additionally, on [Redacted] you can delete comments.
% This is not possible on Namebook.

\subsection{Parallax and Provenance}

\emph{Parallax} is a realtime group chat application that demonstrates
how, under total reification, it is possible for \emph{every} user to employ a different
moderation scheme.
Specifically,
every user, from their own perspective,
is the sole administrator of \emph{all} group chats (that they know about) with
unilateral control over each group's name and membership.
The messages a user sends in a group can only be seen by the users they explicitly
put on that group's membership list.
However, users can also see the changes that other users
make to their own ``views'' of a group and are given the option
to \emph{voluntarily} incorporate those changes,
as shown in Figure~\ref{case-studies:fig:splitsignal}.

Under the hood, a group is represented by a random identifier,
generated when the group is created, and also used as the group's channel.
Changes to a group's name are similar to name changes on Namebook, only they
\texttt{describe} the channel identifier.
Changes to the group's membership are \texttt{"Add"}
and \texttt{"Remove"} activities that \texttt{target} the relevant actor.
Messages are much like posts on Namebook, only they have \texttt{allowed} lists, which are determined
by aggregating the user's own \texttt{"Add"} and \texttt{"Delete"} activities.

Of course, complete independence is not always desirable:
work usually done by just one group administrator must, in Parallax,
be done by every single group user.
We use Parallax to demonstrate an extreme, but by changing X lines of functional code,
the application can be transformed into a more reasonable (but more restrictive) application called \emph{Provenance},
where a group's administrator is the  \emph{creator} of the group chat.

% Provenance determines the \texttt{allowed} list on a user's messages
% by aggregating the \texttt{"Add"} and \texttt{"Remove"} activities made by the first
% actor to post in the chat.

Parallax and Provenance both interoperate.
Some messages sent from one will not be seen in the other
according to their unequal membership lists. However, this messiness is already present and tolerable
in messaging applications like Signal, where users can block other group members,
and email, where any reply can be sent to a different set of recipients.
There is exciting work to be done learning what interfaces make
Graffiti's inevitable assymetry most accessible and engaging.

% While the previous case study demonstrates
% how a user might choose different settings
% \emph{between} applications.
% Here there are settings \emph{within} the application.

% This and the following app are extreme.
% In many cases the user would not to
% have fine grained control over group
% membership and would like to delegate
% some of that work to, say, the person
% who created the groupchat.
% Additional layers like this are possible
% but it is always possible for an application
% like this one to interoperate with
% more traditional ones.

% Maybe show traditional chat app and how
% it interoperates?

\subsection{Wikiffiti}

Wikiffiti is a Wikipedia-like application that demonstrates that
collaborative editing in Graffiti is possible,
even though an actor can only mutate their own objects.
Additionally, unlike Wikipedia,
every user on Wikiffiti can choose which other users have ``permission'' to edit an article,
\emph{retroactively} undoing edits by unpermissioned users,
as shown in Figure~\ref{case-studies:fig:wikiffiti}.
% total reification provides every user with affordance

Edits to each Wikiffiti article are published to
the channel represented by the article's title.
This allows for basic search and, like on Wikipedia,
``disambiguation'' pages can serve as manual search indexes when necessary.

Edits are published and composed together according to Logoot,
a conflict-free replicated data type (CRDT)~\cite{logoot,crdts}.
Logoot, and CRDTs in general, were developed for asynchronous collaborative editing
but, luckily for us, Logoot produces reasonable results when
some edits are ``dropped,'' as we do here intentionally.
Currently, our implementation is inefficient with a 40x space blowup;
however, there are plenty of low-hanging optimizations that one could apply
and release as part of a standard collaborative editing library.

Like Parallax, Wikiffiti is an extreme. Clearly, not every user
has the expertise or desire to vet all the editors of
every article they read. In reality, the work of approving editors
or individual edits will be delegated to
a hierarchy of user access levels,
friend-of-a-friend networks of trust,
or automatic vandalism detectors, for example.
Still, the data underneath can always be reinterpreted, allowing
for new systems to independently evolve that,
for example, might be more welcoming to newcomers~\cite{wikibourgeoisie, wikirisedecline}
or promote edits made by women and non-binary people~\cite{wikigender}.
An application could even highlight edits that are vandalism to some,
but art to others: graffiti.


% Vandalism to some may be empowering to other.

% All the while, maintaining the dissonance that what is vandalism to some,
% may be art to others.
