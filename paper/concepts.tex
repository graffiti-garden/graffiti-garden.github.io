\section{Design Concepts}
\label{concepts}

This section outlines the primary concepts that make up the design of Graffiti
according to our requirements.
We follow with Section~\ref{api}, which outlines the \emph{specific} API that is built out of these
concepts.

At a high level, the Graffiti API works as follows.
\emph{Actors} (basically users) create \emph{objects} which can represent
social artifacts like posts or \emph{activities} like ``likes.''
Objects are posted to a set of \emph{channels} and other users can
discover an object by querying one of its channels. Channels are a flexible representation
of different \emph{contexts} which may include identities, topics, locations,
and more.
Actors cannot modify other actors' objects and so for actors to interact with each others' content,
those interactions must be \emph{reified} into individual
activities,
allowing for different interpretations of them.
Objects can be configured to be either public
(to anyone observing its channels),
or private to a set of users on its \emph{allowed list},
an additional affordance for context management.

\subsection{Objects}
\label{concepts:objects}

Objects are the atomic units that encapsulate \emph{all}
of the social data in Graffiti, including
both social artifacts (such as posts and profiles) and social ``activities'' (such as likes and follows).
Each Graffiti object includes some structured metadata,
such as the actor who created the object,
but an object's ``value'' can be \emph{any} valid JSON object~\cite{json}.
For example, a blog post object may have the value:

\begin{minted}{javascript}
{
  title: "My First Post",
  content: "Hello, world!",
  published: 1611972000, // The time in seconds
}
\end{minted}

This idea of encapsulating social data in JSON objects is
taken directly from the ActivityStreams standard~\cite{activitystreams},
and many common \emph{properties} for objects,
including the ones above, are defined in the
ActivityVocabulary~\cite{activityvocab}.
However, we choose not to inherit the ``linked data'' piece
of these specifications which makes it complex to create new object properties.
Instead, if a developer wants to make a recipes app, for example,
they can freely add an \texttt{ingredients} property to their objects.

We are leveraging an idea here called a \emph{folksonomy}~\cite{folksonomy}.
Social applications
benefit from network effects, and so we expect that developers will make use of
existing properties where possible,
but will introduce new properties when necessary.
The result will inevitably be messy and inconsistent, but, as with the evolution of
particular hashtags, we expect there will be an effective balance of creativity and convergence.
Importantly, the barrier to participating in the folksonomy is low
and does not require a precise ontological agreement,
which some argue contributed to the failure of the semantic
web~\cite{semanticwebtwodecades}.
The folksonomy approach encourages autonomous extensibility,
per Requirement~\ref{requirements:autonomous-extensibility}.

Graffiti objects also have URLs that
start with a \emph{scheme}, just like URLs on the web,
enabling objects to be served by different \emph{parallel} implementations,
per Requirement~{\ref{requirements:parallel-implementations}}.
For example, a post might have the URL
\texttt{graffiti:\allowbreak{}remote:\allowbreak{}pod.\allowbreak{}graffiti.\allowbreak{}garden/\allowbreak{}123},
where the scheme \texttt{graffiti:\allowbreak{}remote:} refers to an implementation that we describe
in Section~\ref{above-and-below:remote-protocol}
instead of \texttt{https:} as seen on the web.
URLs allow the object to be directly fetched%
% \footnote{
%   In Section~\ref{above-and-below} we describe how fixed URLs
%   are still compatible with ``data portability,'' allowing
%   a user to move their data to a new place or protocol without
%   changing the URL.
% }
, but also pointed to, as in the
``like'' activity value below.

%DK ATP: there's a trick I think medium uses that we might want to steal.   they create urls from the title of a post, but *also* include in the url a weired number that I assume is a nonce.   If you change the title of your post, it changes the title part of the url but not the nonce.   Which means that you can look up the post using either the old or the new url and both will work because the medium server is smart enough to look only at the nonce to decide what content is wanted.   Could we do something similar, so that if the user later decided to change the delivery protocol it would possible to interpret old urls (possibly with some help from the user's profile) in a way that would usually still let you fetch the contents?   For example, it would be nice of the user's profile could store a "redirect rule" explaining how to generate the current url from the old

\begin{minted}{javascript}
{
  activity: "Like",
  target: "graffiti:remote:pod.graffiti.garden/123",
}
\end{minted}

This like activity is an example of \emph{reification},
a technique frequently used in the semantic web community~\cite{rdfprimer}.   We use reification to represent actions on data as data itself.
Rather than a post object having a ``like'' method that mutates metadata about the post,
the ``like'' here is a separate object that points to the post object.
Reification allows for new interaction mechanisms to be introduced as easily as new properties,
without changing the underlying system.
While AcitivityStreams and other systems use reification,
Graffiti employs \emph{total reification} where
all interactions, including moderation actions, are reified,
which we discuss in
Section~\ref{concepts:total-reification}.

Finally, objects are mutable, meaning they can be changed or deleted.
Mutability is necessary for the system to be forgiving, per Requirement~\ref{requirements:forgiving}.
The question of \emph{who} can mutate the objects
will be answered once we discuss identity, next, and then total reification.

In summary, Graffiti objects are:

\begin{enumerate}
\item
Flexible: Objects can take on any value according to a self-describing folksonomy.
\item
Globally-Addressable: Every object has a globally unique URL that can be used to fetch or point to it.
\item
Mutable: Objects can be changed or deleted.
\end{enumerate}

\subsection{Actors}
\label{concepts:actors}

An actor is an account that a user can ``log in'' as
to identify themselves, publish objects, and access private objects,
so long as they are on that private object's ``allowed'' list
(to be discussed in Section~{\ref{concepts:allowed-lists}}).
The actor is represented by a globally unique string or \emph{URI},
just as an email account is represented by an email address.
The term \emph{actor} is again inherited from ActivityStreams,
and is used rather than \emph{user}, since a user may own
multiple actors or
an actor may be a non-human entity
like a bot or recommendation service.
All Graffiti objects are associated with an actor, the actor
that created the object, but anonymous
or meronymous~\cite{meronymous} interactions are possible by
creating ``throw away'' actors.

Unlike an ActivityStreams actor, a Graffiti actor does not embody
%DK there is no information that must be associated with an actor other than...
any identifiable information other than its
URI.
Properties like a display name or pronouns are instead
defined in separate objects.
For example, if Alice's actor URI is
\texttt{https://\allowbreak{}alice.\allowbreak{}example.\allowbreak{}com}, a profile object value for her is:
\begin{minted}{javascript}
{
  name: "Alice",
  pronouns: "she/her",
  describes: "https://alice.example.com",
}
\end{minted}
%DK how do you prove Alice owns the actorId; prevent someone else from pretending to?  Does the actorid come with a key alice can use to sign her profile object?

This separation allows for new profile
properties to freely evolve
and it also allows people to have different profiles for different contexts.
For example, a trans person who is not publicly ``out'' may want to
change their name within their queer community without
changing their name to the broader public.
Alternatively, the separation allows users to create nicknames
or ``petnames'' for each other~\cite{petnames}.

%DK there's some inaccurate implication here: while channels can provide unlisted groups (soft access control) there is no mechanism for specifying that access is only *permitted* (hard access control) by actors that are defined to be members of a particular group.
%ATP Or maybe I should call it an asymmetry?   If we really committed to capability based access control, we wouldn't use hard access control by actor id.   instead we would allow an actor to craft a "key" an specify that certain objects are unlocked by that "key" and anyone with the key has access to those objects including to mutate them?

\subsection{Total Reification}
\label{concepts:total-reification}

Within Graffiti, not some but \emph{all} interactions between actors
are \emph{reified}.
This \emph{total reification} allows different applications to have
different ``rules'' regarding what users are ``permitted'' to do, while
still interoperating.

For example, suppose a developer would like to create an application
that interoperates with an existing application,
but that allows certain actors (moderators) to remove other actors' posts.
If the moderators' removals are reified into \texttt{"Remove"} activities,
then the original application can simply ignore the activities
while the new application can attend to the new signal and hide ``removed'' posts from the display.
To be clear: an offending object is not removed from Graffiti as a whole,
it is just removed from any application that understands \emph{and respects}
the moderators' \texttt{"Remove"} activities.%

These moderation rules can evolve as developers
introduce new objects and properties,
or new procedures to interpret those objects.
For example,
perhaps the set of moderators is elected by actors who vote with reified \texttt{"Vote"} activities.
Or, perhaps an application hides posts that exceed a certain threshold of \texttt{"Remove"} activities,
regardless of whether the removing actors are considered moderators.
Applications may also consider authors of posts to be the \emph{de facto} moderators of the replies to their
own posts or allow them to disable replies altogether by adding a \texttt{disableReplies}
property to their post objects.
A fully democratic system that allows actors to introduce and ratify new moderation proposals,
as described by the PolicyKit project~\cite{policykit}, could also be reified into
individual-actor activities.
Importantly, all of these systems can coexist in different interoperating
applications and users can freely opt in or out of different systems simply by switching
the application they use.

The flexible object values that we described in Section~\ref{concepts:objects} make arbitrary reified interactions \emph{possible}, but we use the term \emph{total} reification because in Graffiti this is the \emph{only} way to represent interactions.
We impose this rule to achieve the \emph{adversarial} part of our adversarial interoperability requirement (\ref{requirements:adversarial-interop}).
If an application could modify the \emph{actual} state of an object, then it could force all other applications to accept its decisions about the object.
For example, if an actor could actually \emph{delete} another actor's post
from Graffiti \emph{as a whole}
then it would become impossible to build
an application that does \emph{not} recognize the deleting actor as a moderator.
Thus, we ensure that:
\begin{enumerate}
\item
Objects are \emph{proprietary}: Only the creator of an object can modify or delete it.
\end{enumerate}
Conversely, adversarial interoperability means that no applications should be able to ``block'' another from applying an interaction, which means that it must be possible for an actor to publish
their reified action object somewhere that is visible
to relevant applications.
Graffiti objects are published in channels, described in the next
section, and therefore:
\begin{enumerate}
\setcounter{enumi}{1}
\item
Channels are \emph{permissionless}: actors can freely read from and write to channels that they know of.
\end{enumerate}

\subsubsection{Antipaternalism}

Total reification grants people the ability to keep content off of ``their'' applications,
but it prevents anyone from keeping content off of Graffiti as a whole\footnote{%
It \emph{is} possible for content to be deleted in implementations ``below'' the API,
allowing hosting providers to delete illegal content,
like Child Sexual Abuse Material (CSAM).
However, if the implementation is end-to-end encrypted,
it may not be possible for the hosting provider to
detect illegal content,
a fundamental dilemma of encryption technology.
}.
While this antipaternal stance may seem extreme,
it is no different than the web or other open infrastructures,
like ActivityPub or the AT Protocol:
people can always find \emph{some}
web server, ``instance,''
or ``personal data store'' to host their content.

However, our design is more than a reluctant acceptance of the way things are.
In line with harm reduction practices~\expandafter{\expandafter\cite{harmreduction}},
there are benefits to keeping harmful content---which will inevitably
exist---within Graffiti as a whole, while filtered from mainstream spaces.
For example, when users drawn into extremist ideologies are not
siloed away on entirely disconnected systems
like 8kun or Gab~\cite{8kun,gab}, it becomes possible to create off-ramps
to help them disengage gradually without going ``cold turkey.''
Of course, further work is needed to assess the efficacy of
such a strategy and its possible harms.

\subsubsection{Accountability}

Total reification works to enforce ``adversarial interoperability''
per Requirement~{\ref{requirements:adversarial-interop}}
and thus to protect users from leaders---even of small communities---that abuse their role as
``benevolent dictator for life''~{\cite{governablespaces}}.
Our solution matches the suggestion given by Melder et al.
in their study of Fediverse communities with imperfect leadership
to introduce a ``user-level selective ability to bypass [moderator] filters''~{\cite{blocklistboundary}}.
However, it is worth considering whether total reification makes it so easy
to opt out of a community's norms that users are not held accountable for their actions.

While there is work to be done studying this in the context of Graffiti specifically,
work by Hessel et al.
suggests the opposite effect~{\cite{highlyrelatedcommunities}}.
They find that users who explore parallel community spaces
with different norms not only still meaningfully participate in their original community space,
but also become more engaged in it.

\subsection{Channels}
\label{concepts:channels}

Channels are the mechanism Graffiti uses to notify
actors of new objects.
Rather than sending objects directly to specific actors\footnote{
    See our discussion of ActivityPub's ``Actor Model'' in Section~\ref{related-work:activitypub}.
},
channels are a publish-subscribe mechanism~\cite{pubsub}
where objects published to a channel
can be read by actors subscribing to that channel.
However, unlike most publish-subscribe systems,
an object remains ``in'' a channel until
it is explicitly removed from that channel
or the object as a whole is deleted.
Therefore, we can consider a channel both a ``place''
an object can be and a property of the object itself.

The channel namespace is \emph{global} with every string (up to a certain length)
naming exactly one channel.
Like object properties, these channel names have no inherent meaning
other than the meaning ascribed to them by developers or users.
In practice, the meaning of a channel is a particular social \emph{context} and the channel
name is some sort of \emph{shared information} implicit to that context.
For example, the channel named by an object's URL is a natural place to post content ``\emph{talking about}'' that object, like replies, and we expect that most applications will follow that norm.
Importantly, content within channels is effectively hidden from applications
that do not know their names,
an affordance to prevent context collapse, per Requirement~\ref{requirements:context-differentiation}.

We will first illustrate the power of channels
to model different types of \emph{replies}.
Then, we describe the formal properties of channels and their other uses.

\subsubsection{Replies}

Most social applications allow users to reply to posts, but there are surprisingly subtle
contextual differences between different reply designs.
For example, on Instagram, replies to a post are only ever displayed to people viewing that particular post\footnote{
Instagram used to have a ``Following'' tab where a user could see the replies that people they follow had posted across the application.
This led to numerous examples of context collapse and so Instagram discontinued it in 2019~\cite{instagramfollowingtab}.
}, while on X, a user's replies from across the application can all be found in their Replies tab.
These differences reflect the platforms’ contrasting design goals of intimacy versus reach.

With channels, \emph{both} designs can \emph{co}exist:
\begin{itemize}
\item
To create an Instagram-like reply,
an application should post the reply to the channel
named by the original post's URL---for brevity, the ``post's channel.''
Applications displaying the original post implicitly know that post's channel name
and can query the channel to populate the reply thread.
\item
To create an X-like
reply, an application should post the
reply to \emph{both} the post's channel
and the channel named by the replier's actor URI---for brevity, the ``replier's channel.''
Applications displaying the replier's profile implicitly know that replier's
channel name and can query the channel to populate the Replies tab.
\end{itemize}

Finally, what happens if the reply is \emph{only} posted to the
replier's channel?
In that case, the reply becomes like a Quote Tweet\footnote{
Prior to 2020. \url{https://x.com/Support/status/1300555325750292480}
}: the reply is visible to the replier's followers but
not the audience of the original post, as they do not have implicit
access to the replier's channel.
A summary of these different designs is shown in Table~\ref{concepts:channel-replies}.

\begin{table}[htbp]
\centering
\caption{Reply designs according to channel usage}
\label{concepts:channel-replies}
\begin{tabular}{cc|c|c}
\multicolumn{4}{l}{\emph{Is the reply posted to the...}} \\
& \multicolumn{1}{c}{} & \multicolumn{2}{c}{\emph{\textbf{...post's channel?}}} \\
& & {\checkmark} & {$\times$} \\
\cline{2-4}
\multirow{2}{*}{\shortstack{\emph{\textbf{...replier's}} \\ \emph{\textbf{channel?}}}}
& {\checkmark} & X Reply & Quote Tweet \\
\cline{2-4}
& {$\times$} & Instagram Reply & N/A
\end{tabular}
\end{table}

Importantly, these reply designs interoperate in a way that allows content producers to target a particular audience,
regardless of the applications others choose to use:
Instagram-like replies will not be displayed in X-like Replies tabs.

Therefore, channels can be considered a constraint on application personalization,
in tension with our overarching Requirement~\ref{requirements:personalization}.
We introduce this constraint because, unlike total reification---where applications
can present different interpretations of the underlying
content simultaneously, e.g. a post is ``removed'' in one application but not in another---a reply
cannot be simultaneously ``seen'' and ``not seen'' by a potential audience member.
We prioritize giving posters the ability to express their ``targeted imagined audiences''~\cite{imaginedaudience}
to mitigate context collapse, at the cost of slightly reducing consumer autonomy.

\subsubsection{Properties}

In summary, channels are:

\begin{enumerate}
\item
Persistent: When an actor publishes an object to a channel, other actors can
continue to retrieve the object until it is removed from the channel or deleted.
\item
Global: The channel namespace is global with every string (up to a certain length)
corresponding to exactly one channel.
\item
Crosspostable: An object can be associated with more than one channel (or no channel at all).
\item
Obscure: It is not possible to read from or write to a channel without knowing
its name.
\item
Permissionless: Actors can freely read from and write to any channel \emph{that they know about}.
\end{enumerate}

Our emphasis on the last property, which we introduced in our discussion of
total reification, is due to the additional ``Obscure'' property
which allows channels to provide a form of \emph{security by obscurity}, like an ``unlisted'' link.
However, security by obscurity does not allow for \emph{revocation}\footnote{
    See Section~\ref{related-work:ocaps} for a discussion of how channels relate to ``object capability security''
    which is similar to security by obscurity but allows for revocation.
}
of access,
which is why objects may also be access-controlled,
as we will discuss in Section~{\ref{concepts:allowed-lists}}.

Despite the fact that the ``Obscure'' property implies that
there is no global directory of channels,
many channels may be found by spidering Graffiti like a search engine.
Therefore, a determined
developer could make an application that intentionally collapses contexts.
This is an unfortunate possibility with any public content.
For example, snoopreport.com sells a context-collapsing service for Instagram.

Given this, users with privacy critical needs should, again, use the
access control affordances described in Section~{\ref{concepts:allowed-lists}}.
For less critical needs,
we hope to cultivate social norms that dissuade scraping, similar to informal notions of privacy in the real world.
For example, it is entirely possible for a person in the real world
to attend various support groups, like alcoholics anonymous,
just to spread people's secrets, but doing so would be shameful.
On the web, \texttt{robots.txt} is a functional socially-enforced standard
to prevent web scraping.


\subsubsection{Other Channel Meanings}

We have seen how channels can represent \emph{people} and \emph{posts}
which cover the contextual needs of most microblogging applications,
but channels can also represent media, topics, websites, and locations,
and more meanings can be freely introduced in an evolving folksonomy~\cite{folksonomy}.
A table of possible channel meanings and the types of applications
those meanings can enable is shown in Table~\ref{concepts:channels-and-applications}.

\begin{table}[h]
\small
\caption{Applications enabled by channel meanings}
\label{concepts:channels-and-applications}
\centering
\begin{tabular}{|l|l|}
\hline
\textbf{Channel Meanings} & \textbf{Representative Applications} \\ \hline
External Media & Letterboxd, GoodReads, Genius \\ \hline
User Media & YouTube, Medium, Github \\ \hline
Topics & Reddit, Discord, Slack \\ \hline
Collections & are.na, Pinterest, Spotify playlists \\ \hline
Websites & Hypothes.is, Pinboard \\ \hline
Physical Locations & Tinder, Craigslist, Nextdoor \\ \hline
Virtual Locations & Second Life, Minecraft \\ \hline
Events & Partiful, Facebook Events \\ \hline
Products/Services & Amazon reviews, Yelp \\ \hline
Documents & Wikipedia, Google Docs \\ \hline
\end{tabular}
\end{table}

In some cases, an existing standard can be used as a channel
name, like the URL of a website, for Hypothes.is analogs,
or the ISBN of a book, for GoodReads analogs.
In other cases, the channel name may be user generated,
as with public contexts like subreddits or Wikipedia pages,
%DK as discussed, it is problematic to use human-readable names for subreddits, since it means you can never change them.   and while i can see some limited benefit to human-readable wikipedia names, i don't think it extends to subreddits.   Much better to have a directory page of subreddits and their corresponding channels.
or randomly generated for ``unlisted'' contexts like
Discord channels, Pinterest boards, or Google documents.
We discuss how continuous contexts, like geographic locations, can be
mapped onto specific channel names in Appendix~\ref{appendix:more-channel-names}.

\subsection{Allowed Lists}
\label{concepts:allowed-lists}

Allowed lists are an access control mechanism
that an actor can optionally apply to an object
to limit its visibility to a specific list of actors.
Allowed lists enable direct messaging and private groups,
as well as ``visibility control''~\cite{visibilitycontrol, toogayforfacebook}
affordances in networked settings such as
Facebook-like ``Friends''-only posts.

We use this simple whitelist mechanism of access control rather than,
say, Unix file types, because the additional affordances that
such a system would provide can be built more flexibly with total reification.
For example, in Section~{\ref{case-studies:parallax}} we show how mutable group
membership in a messaging application can be constructed.

Allowed lists do not replace channels when they are used
but rather complement them,
with allowed lists providing guaranteed privacy
and channels specifying the context in which the content should appear.
For example, a friend group may communicate via a group chat,
plan a trip via a private spreadsheet, and post ``Close Friend'' stories
for each other to see.
Objects representing interactions in these different scenarios
would have identical allowed lists,
but objects would be placed in different channels based on their context.
