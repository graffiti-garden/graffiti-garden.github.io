% \section{Design Overview and Concepts}
\section{Design Concepts}
\label{concepts}

This section outlines the primary concepts that make up the design of Graffiti
according to our requirements.
We follow with Section~\ref{api}, which outlines the \emph{specific} API that is built out of these
concepts.

At a high level, the Graffiti API works as follows.
\emph{Actors} (basically users) create \emph{objects} which can represent
social artifacts like posts or \emph{activities} like ``likes.''
Objects are associated with a set of \emph{channels} and other users can
discover an object by querying one of its channels. Channels a are flexible representation
of different \emph{contexts} which may include identities, topics, locations,
and more.
Actors can not modify other actors' objects and so for actors to interact with each others' content
those interactions must be \emph{reified} into individual activities,
allowing for different interpretations of the interactions.

% Our objects and actors inherit from the
% ActivityStreams standard~\ref{activitystreams}
% at a high level our concepts \emph{total reification}
% and \emph{channels} are our own.

\subsection{Objects}
\label{concepts:objects}

Objects are the atomic units that encapsulate \emph{all}
of the social data in Graffiti, including
both social artifacts (such as posts and profiles) and social ``activities'' (such as likes and follows).
Each Graffiti object includes some structured metadata,
such as the actor who created the object,
but an object's ``value'' can be \emph{any} valid JSON object.
For example, a blog post object may have the value:

\begin{minted}{javascript}
{
  title: "My First Post",
  content: "Hello, world!",
  published: 1611972000, // The time in seconds
}
\end{minted}

This idea of encapsulating social data in JSON objects is
taken directly from the ActivityStreams standard~\cite{activitystreams},
and many common \emph{properties} for objects,
including the ones above, are defined in the
ActivityVocabulary~\cite{activityvocab}.
However, we choose not to inherit the ``linked data'' piece
of these specifications which makes it complex to create new object properties.
Instead, if a developer wants to make a recipes app, for example,
they can freely add an \texttt{ingredients} property to their objects.

% Graffiti application developers can freely introduce new properties as necessary
% as part of an evolving \emph{folksonomy}~\cite{folksonomy}.
% We leverage the fact that JSON can be \emph{self-describing}
% --- a message can contain structured data with arbitrary properties and
% values that are irrelevant and opaque to the delivery infrastructure but
% meaningful to specific applications.
% For example, the developer of an application for sharing recipes can freely introduce a new
% \texttt{ingredients} property to objects.
% allowing them to benefit from pre-existing data
% made in other applications
We are leveraging an idea here called a \emph{folksonomy}~\cite{folksonomy}.
Social applications
benefit from network effects, and so we expect that developers will make use of
existing properties where possible
but will introduce new properties when necessary.
The result will inevitably be messy and inconsistent, but, as with the evolution of
particular hashtags, we expect there will be an effective balance of creativity and convergence.
Importantly, the barrier to participating in the folksonomy is low
and does not require a precise ontological agreement,
which some argue contributed to the failure of the semantic
web~\cite{semanticwebtwodecades}.
The folksonomy approach encourages autonomous extensibility
as per requirement \ref{requirements:autonomous-extensibility}.

% This folksonomy is an effective mean of ensuring that objects are autonomously extensible.
% We also provide tooling that allows users to filter data and provide human-readable
% fallbacks for data they can't parse.

Graffiti objects also have URLs that
start with a \emph{scheme}, just like URLs on the web,
enabling objects to be served by different \emph{coexisting} protocols.
For example, a post might have the URL
\texttt{graffiti\allowbreak:remote\allowbreak:pod\allowbreak.graffiti\allowbreak.garden\allowbreak/123},
where the scheme \texttt{graffiti:remote:} refers to a protocol we describe
in Section~\ref{above-and-below:remote-protocol}
instead of \texttt{https:} as seen on the web.
URLs allow the object to be directly fetched\footnote{
  In Section~\ref{above-and-below} we describe how fixed URLs
  are still compatible with ``data portability,'' allowing
  a user to move their data to a new place or protocol without
  changing the URL.
}, but also pointed to, as in the
``like'' activity below.
% enabling other users to ``like'' that post by creating a new object with the value:

\begin{minted}{javascript}
{
  activity: "like",
  target: "graffiti:remote:pod.graffiti.garden/123",
}
\end{minted}

This like activity is an example of \emph{reification},
a technique frequently used in the semantic web community~\cite{rdfprimer},
which permits actions on data to be represented as data themselves.
Rather than a post object being an object that has a ``like'' method,
the ``like'' here is a separate object that points to the post object.
Reification allows for new interaction mechanisms to be introduced as easily as new properties,
without changing the underlying system.
While AcitivityStreams and other systems use reification,
Graffiti employs \emph{total reification} where
all interactions, including moderation actions, are reified,
which we discuss in
Section~\ref{concepts:total-reification}.

Finally, objects are mutable, meaning they can be changed or deleted,
Mutability is necessary for the system to be forgiving per requirement~\ref{requirements:forgiving}.
The question of \emph{who} can mutate the objects
will be answered once we discuss identity next and then total reification.

In summary, Graffiti objects are:

\begin{enumerate}
\item
Flexible: Objects can take on any value according to a self-describing folksonomy.
\item
Globally-Addressable: Every object has a globally unique URL that can be used to fetch or point to it.
\item
Mutable: Objects can be changed or deleted.
% we add the two following contraints on objects
% \item
% Proprietary: Only the creator of an object can modify or delete it.
% \item
% Permissioned: Objects can have an allow list of users who can view them.
% \item
% Ducktyped: Objects have no type.
\end{enumerate}

% Mutable, proprietary, and permissioned are necessary for the right to be forgotten.
% \item
% Objects can be associated with one or more channels.
% \item
% You can either get an object by one of its channels or by its URI.

\subsection{Actors}
\label{concepts:actors}

An actor is an account that a user can ``log in'' as
to identify themselves, publish objects, and receive ``private'' objects.
The actor is represented by a globally unique string or \emph{URI},
just as an email account is represented by an email address.
The term \emph{actor} is again inherited from ActivityStreams
and is used rather than \emph{user} since a user may own
multiple actors or
an actor may be a non-human entity
like a bot or recommendation service.
All Graffiti objects are associated an actor, the actor
that created the object, but anonymous
or meronymous~\cite{meronymous} interactions are possible by
creating ``throw away'' actors.

% We also use the actor almost synonymously with a user,
% In most cases an actor can be considered synonymous with a user,
% however, like an email address, a user can own multiple actors
% or actors may be non-human entities like bots or recommendation services.
% The term \emph{actor} was originally used for this purpose
% in the ActivityPub standard~\cite{activitypub}.

% A user can ``log in'' to an actor and then publish objects as that actor.
% In fact, it is not possible create objects without an actor,

% a person can create ``throw away'' actors.
% It is not possible for one user to impersonate another user's actor.
% Importantly, to enable interoperability
% as required by \ref{requirements:adversarial-interop},
% a Graffiti actor is not tied to any particular application so that users can freely move between
% different applications.

Unlike an ActivityStreams actor, a Graffiti actor does not embody any identifiable information other than its
URI.
Properties like a display name or pronouns are instead
defined in separate objects.
For example, with Alice's actor URI in the \texttt{describes}
property, a profile object for her is:
\begin{minted}{javascript}
{
  name: "Alice",
  pronouns: "she/her",
  describes: "https://alice.example.com",
}
\end{minted}

This separation allows for new profile
properties to freely evolve
and it also allows people to have different profiles for different contexts.
For example, a trans person who is not publicly ``out'' may want to
change their name within their queer community without
changing their name to the broader public.
Alternatively, users could create nicknames
or ``petnames'' for each other~\cite{petnames}.
% Pet names

Actors are also used for privacy.
Objects can be \emph{access-controlled} and only
made visible to a limited set of actors, enabling
direct messaging.
Forms of privacy that do not revolve around identity
like ``unlisted'' groups are handled
with the \emph{channels} concept we discuss in
Section~\ref{concepts:channels}.

% For example, an actor does not includ a
% name, profile picture, or biography associated with it.
% Instead, profiles are defined by seperate profile \emph{objects}
% (described in the following Section~\ref{concepts:objects}),
% allowing applications to freely introduce new profile
% metadata or for users to have multiple profiles for different contexts,
% making profiles autonomously extensible as required by \ref{requirements:autonomous-extensibility}.
% Objects and their relation to autonomous extensibility are described next.
% \item
% Access-Controllable: Objects can be made private and only visible to a limited set of actors.


\subsection{Total Reification}
\label{concepts:total-reification}

Within Graffiti, not some but \emph{all} interactions between actors
are reified.
This \emph{total reification} allows different applications to have
different ``rules'' regarding what users are ``allowed'' to do while
still interoperating.
% This allows developers to freely introduce new interactions
% per our requirement for autonomous extensibility (\ref{requirements:autonomous-extensibility}),
% without breaking interoperability.

% We define interaction relativity to mean that ``interaction between two individuals only
% exists relative to an application'' or equivalently that not some but \emph{all} interactions
% are \emph{reified}.
% For example, a like is not a method on an object.
% So if an application does not underatnd the action and users don't see it, did the like really occur?

% if a person likes another person's post

% For example if an application \emph{A} is built to understand like objects,
% \emph{A} may tally up all like objects that point
% at a post and display the like count next to the post.
% Alternatively, an application \emph{B} may give poster the ability disable likes
% by adding a \texttt{disableLikes} property to their post objects
% causing \emph{B} to not display a like counter.
% A person's like may ``occur'' in \emph{A} but not in \emph{B}.
% From the perspecive of one application the like occurred and from the
% perspective of another it did not.
% Without applications to interpret the data, the post and like objects
% are disconnected and meaningless.
% Whether a user has liked another users post is relative to which
% application they use.

% The introduction and interpretation of new properties and objects
% that clarify whether an interaction has, or has not, occured
% is entirely up to the application developers themselves.

% Total reification is powerful when applied to moderation.
For example, suppose a developer would like to create an application
that interoperates with an existing application,
but that allows certain actors (moderators) to delete other actors' posts.
If the moderators' deletions are reified into \texttt{"delete"} activities,
then the original application can simply ignore the activities
while the new application removes ``deleted'' posts from the display.

These rules can evolve as developers
% Moderation can evolve even further under total reification
introduce new objects and properties
or new procedures to interpret those objects.
For example,
perhaps the set of moderators is elected by actors who vote with reified \texttt{"vote"} activities.
Or, perhaps an application removes posts that exceed a certain threshold of delete activities,
regardless of whether the deleting actors are considered moderators.
Applications may also consider authors of posts to be the defacto moderators of the replies to their
own posts or allow them to disable replies altogether by adding a \texttt{disableReplies}
property to their post objects.
A fully democratic system that allows actors to introduce and ratify new moderation proposals,
as described by the PolicyKit project~\cite{policykit}, could also be reified into
individual actor activities.
Importantly, all of these systems can coexist in different interoperating
applications and users can freely opt in or out of different systems simply by switching
the application they use.
% can evolve independently and
% users can freely opt in or out of different
% procedures because all of the underlying interactions are reified.

% The introduction and ratification of new proposals

% This is similar to the concept of ``stackable moderation'' proposed by Bluesky~\cite{bluesky}
% but more general.
% In stackable moderation, users can choose which moderators they would like to filter their content.
% In interaction relativity, the application developer can choose which moderators they would like to filter their content.
% This is more general because the application developer can choose to filter content based on any reified activity,
% not just moderation actions.
% For example, day that a Moderator wants to delete another user's post.
% (Reference "issues with mastodon paper" which cites moderation issues
% locked into particular places, work redone, etc.)


% This helps with that paper that eric gilbert sent. Moderation as data.
% For example say that Alice wants to delete a post that Bob made.
% For moderation this is like composable moderation. But also more.
% For example, one application could have a single fixed moderator,
% another could allow users to choose which moderators they would like filter their content
% like [Bluesky's stackable moderation](https://bsky.social/about/blog/03-12-2024-stackable-moderation),
% and another could implement a fully democratic system like [PolicyKit](https://policykit.org/).
% Each of these applications is one interpretation of the underlying refieid user interactions and
% users can freely switch between them.

% Another example is collaboration. Say one user would like to modify
% Say a user would like to edit another user's post.
% a document but another would like to keep it the same. CRDTs.

Having the flexible object values that we described in Section~\ref{concepts:objects}
makes reifiecation \emph{possible} but to be ``totally reified,''
reification must be \emph{necessary} for interactions between
actors to occur.
For example, it should not be possible for an actor to \emph{actually}
delete another actor's post because that makes it impossible to build
an application that does \emph{not} recognize the deleting actor as a moderator.
That situation would violate the ``adversarial'' part of ``adversarial interoperability,''
from requirement~\ref{requirements:adversarial-interop}.
As such, we ensure that:
\begin{enumerate}
\item
Objects are \emph{proprietary}: Only the creator of an object can modify or delete it.
\end{enumerate}
% Conversely, actors should not be restricted from publishing new
% activities.
% as this limits the
% from creating new objects and activities,
% as this limits the extensibility of the system.
% hinder the creation of new reified activities.
% introduction of new interactions.
Conversely, to give users ``permission'' to perform an action,
they must be able to publish
their reified action object somewhere that is visible
to relevant applications.
Graffiti objects are published in channels, described in the next
section, and therefore:
\begin{enumerate}
\setcounter{enumi}{1}
\item
Channels are \emph{permissionless}: actors can freely read from and write to channels.
\end{enumerate}
% All permission structures are built on top of permissionless base
% through reified actions and application interpretation.

Our case studies in Section~\ref{case-studies} demonstrate
some of the new and interesting interactions within and between applications
that are possible with total reification.
We also are concurrently writing about its social
ramifications: its annihilation of any ``paternalistic''
rule systems and also its great potential for empowerment.

% The fact that the API does not allow content to \emph{actually} be deleted
% has consequences for illegal content, like CSAM.
% However, it is possible for some implementations under the API to detect
% and prevent this content which we discuss in both
% Section~\ref{protocols} and Section~\ref{discussion}.
% There is also some subtlety in how collaborative editing might be done under total reification,
% for systems like Wikipedia or Google Docs,
% but we demonstrate in Section~\ref{case-studies} that
% it is possible with the use of conflict-free replicated data types (CRDTs)~\cite{crdts}.

% which we discuss in our discussion.
% While we argue it is possible to make any social application
% with these properties there are implications for the fact that content
% cannot actualy be deleted of prevented.
% We discuss implications for illegal content, like CSAM, in our discussion.


\subsection{Channels}
\label{concepts:channels}

Channels are the mechanism Graffiti uses to notify
actors of new objects.
Rather than sending objects directly to specific actors\footnote{
    See our discussion of ActivityPub's ``Actor Model'' in Section~\ref{related-work:activitypub}.
},
channels are a publish-subscribe mechanism~\cite{pubsub}
where objects published to a channel
can be read by actors subscribing to that channel.
However, unlike most publish-subscribe systems,
an object remains ``in'' a channel until
it is explicitly removed from that channel
or the object as a whole is deleted.
Therefore, we can consider a channel both a ``place''
an object can be and a property of the object itself.

The channel namespace is also \emph{global} with every string (up to a certain length)
corresponding to exactly one channel.
Like object properties, these channels strings have no inherent meaning
other than the meaning ascribed to them by applications.
In practice, the meaning of a channel is a particular \emph{context} and the channel
string is some sort of \emph{shared information} implicit to that context.
We will first illustrate the power of channels
to model different types of \emph{replies},
preventing context collapse per requirement~\ref{requirements:context-differentiation}
while maintaining interoperability.
Then, we describe the formal properties of channels and their other uses.

% as part of
% an evolving folksonomy.
% In practice we will see that they can represent
% some sort of \emph{shared information},
% that serves as a point of connection between two actors.
% They can represent some sort of \emph{shared information}
% that two actors have and therefore represent a particular \emph{context}

% However, we will see they
% However, we will see they are a powerful way to represent
% \emph{context} and therefore to prevent \emph{context collapse},
% as per requirement~\ref{requirements:context-differentiation}.
% To illustrate this we will examine the case of \emph{replies}.


% However, network effects provide incentives for application
% developers to establish shared conventions.
% We explore below
% However this meaning

% (Channels are in a global namespace.)
% Like with object properties, channels represent a \emph{folksonomy}.
% They have no inherent meaning other than the meaning ascribed to them
% by particular application.
% As we mentioned no one owns channels,
% as other ``rules'' are added on top from total reification.

% One of the primary benefit of channels is their
% natural ability to represent \emph{context}
% and therefore to prevent \emph{context collapse},
% as per requirement~\ref{requirements:context-differentiation}.
% ~\cite{contextcollapse}.

% Channels are a way of organizing ``where'' objects are.
% Objects have access control which enables private messages,
% however other forms of context collapse can occur when
% content is technically public but only intended for a particular audience.
% Channels are our solution to representing this need in the API,
% as required by \ref{requirements:context-differentiation},
% while still being
% interoperable as required by~\ref{requirements:adversarial-interop}.

% \subsubsection{Replies}

% To understand how channels can work to prevent context collapse,

\subsubsection{Replies}

Most social applications allow users to reply to posts, but there are surprisingly subtle
contextual differences between different reply designs.
For example, currently on Instagram
if you reply to a post, that reply can only be seen by other
people also viewing that particular post.
However, prior to 2019, Instagram had a Following tab where a user could see
the replies that the people they follow had posted across the application.
The Following tab led to a priest outing himself as gay among numerous other
examples of context collapse~\cite{instagramfollowingtab}.
On Twitter/X, it is still possible to see all of a user's replies by going
to their Replies tab.

With channels, it is possible for \emph{both} designs to \emph{co}exist.
% preventing context collapse to those
% and
% the choice
% is up to the \emph{replier} rather than viewers of those replies.
To make an Instagram-style reply,
an application should post that actor's reply to the channel
represented by the original post's URL---for brevity, the ``post's channel.''
Applications displaying the original post implicitly know that post's
URL and therefore,
by convention, they know to query the post's channel for replies to that post.
% for replies but it won't be exposed to the rest of the user's followers.
To make a Twitter/X-style reply, an application should post the
actor's reply to \emph{both} the post's channel
and the channel represented by the replier's actor URI---for brevity, the ``replier's channel.''
An actor's followers all implicitly know the actor's URI and so
again, by convention, their applications know to query the replier's channel
for content related to the actor.

Finally, what happens if the reply is \emph{only} posted to the
replier's channel?
Then the reply becomes like a quote tweet
(prior to 2020\footnote{
    \url{https://x.com/Support/status/1300555325750292480}
})
where the reply is only visible to the replier's followers but
not the audience of the original post.
A summary of these different designs is shown in Table~\ref{concepts:channel-replies}.

\begin{table}[htbp]
\centering
\caption{The reply's channels include the...}
\label{concepts:channel-replies}
\begin{tabular}{cc|c|c}
& \multicolumn{1}{c}{} & \multicolumn{2}{c}{\textbf{...post's channel?}} \\
& & ︎{\emoji✔️} & {\emoji❌} \\
\cline{2-4}
\multirow{2}{*}{\shortstack{\textbf{...replier's} \\ \textbf{channel?}}}
& ︎{\emoji✔️} & Twitter Reply & Quote Tweet \\
\cline{2-4}
& {\emoji❌} & Instagram Comment & N/A
\end{tabular}
\end{table}

Importantly, these reply designs interoperate in a way that provides content producers with
a consistent \emph{audience} regardless of the applications those audience members choose to use.
For example, if an actor posts an Instagram-like reply, that reply will \emph{only} show up
in the context of the original post. An actor using a Twitter-like application
will not see the replying actor's reply in their Replies tab because it simply was not
posted to a channel that the application knows to query.

Channels can be considered as a limit on application personalization,
counter to our overarching requirement.
It is \emph{not} possible for an
application to find all a user's replies if they have not been posted to channels that
the application knows to query.
We establish this limit because, unlike with total reification where it is
possible for application to provide different interpretations of the underlying
content simultaneously---a post is ``deleted'' in one application but not in another---%
it is not possible for an actor's reply to be simultaneously ``seen'' and ``not seen''
by a potential audience member.
We opt to give posters the ability to express their intended audience
to mitigate context collapse, at the cost of a small loss of autonomy for viewers.
% We opt to protect users from context collapse.


% Unlike total reification, which gives autonomy to \emph{see} what they like this is
% a limit on what applications \emph{cannot} see.
% Here the choice of a person to
% It is possible for a user to
% It is not possible to require that a person does not read something and give them the ability to read it.
% Unlike deletes, it is not possible for one person to not
% This is because it is not possible for someone


% A user who posts a reply in an Instagram-like application can be sure that their reply
% will only be surfaced to users viewing that particular post, re
% can be sure that their reply will
% only be seen by

% should expect that their reply will not be leaked

% Importantly context is preserved across interoperation.
% If a user posts their reply in
% an Instagram-like app they should expect that their reply will not be leaked even if
% a user is using Twitter.
% The choice is up to the user rather than

% To make an Instagram-style reply, that reply should be posted to the channel
% containing the original post's URI.
% If we also place the comment in the channel represented by the commenter's URI (their
% {@link GraffitiObjectBase.actor | `actor` URI}), then people viewing the commenter's profile
% will also see the comment, giving it more visibility, like a reply on Twitter.
% If we *only* place the comment in the channel represented by the commenter's URI, then
% it becomes like a quote tweet ([prior to 2020](https://x.com/Support/status/1300555325750292480)),
% where the comment is only visible to the commenter's followers but not the audience
% of the original post.

% Channels embody a publish / subscribe model where
% rather than objects being delivered directly to a particular
% actor, they are published to a channel and other actors can
% subscribe to that channel to receive the objects.
% However most publish subscribe systems are not persistent,
% where as Graffiti objects stay ``in a channel'' until
% they are deleted.

% Additionally, channels are \emph{global}


% Specifically, channels are similar to a publish/subscribe,
% only those systems usually do not persist,
% and to IRC and Slack channels whose
% concept name we use for familiarity.

% They are also \emph{global}

% Like a television channel, users can publish content to a particular channel
% and user watching that channel will receive it.
% Unlike a television channel, Graffiti channels are:

\subsubsection{Properties}

In summary, channels are:

\begin{enumerate}
% \item
% Bidirectional: Actors can both read from and write to channels.
\item
Persistent: When an actor publishes an object to a channel, other actors can
continue to retrieve the object until it is removed from the channel or deleted.
\item
Global: The channel namespace is global with every string (up to a certain length)
corresponding to exactly one channel.
\item
Crosspostable: An object can be associated with more than one channel (or no channel at all).
\item
Obscure: It is not possible to read from or write to a channel without knowing
its string.
\item
Permissionless: Actors can freely read from and write to any channel \emph{that they know about}.
\end{enumerate}

We have clarified the last property, which we introduced in our discussion of
total reification, because of the additional ``Obscure'' property
which allows channels to provide a form of \emph{security by obscurity} like an ``unlisted'' link.
Security by obscurity doesn not allow for \emph{revocation}\footnote{
    See Section~\ref{related-work:ocaps} for a discussion of how channels relate to ``object capability security''
    which is similar to security by obscurity but allows for revocation.
}
of access,
hence why objects are also access-controlled as mentioned in Section~\ref{concepts:actors}
To send someone a direct message that message should be placed in
the receiving actor's channel, where by convention they know to look for it,
and access control on it should be limited to only the recipient.

Many channels, like the ``post's channel'' in our replies example,
can be found by spidering Graffiti like a search engine and so a determined
developer could make an application that intentionally collapses contexts.
While this is a possibility and another reason why access control is necessary for privacy-critical situations, we hope to cultivate social norms that dissuade this behavior, similar to informal notions of privacy in the real world.
For example, it is entirely possible for a person in the real world
to attend various support groups, like alcoholics anonymous,
just to spread people's secrets, but doing so would be shameful.
On the web, \texttt{robots.txt} is functional socially-enforced standard
to prevent web scraping.

% There is also a social component here, where for preventing
% context collapse we're assuming that people are not going out
% of their way to spider all the channels they can to find hidden information.
% Like
% TODO robots.txt

% In some sense, channels provide a sort of social access control by forming
% expectations about the audiences of different online spaces.
% As a real world analogy, oftentimes support groups, such as alcoholics
% anonymous, are open to the public but people in those spaces feel comfortable sharing intimate details
% because they have expectations about the other people in attendance.
% If someone malicious went to support groups just to spread people's secrets,
% they would be shamed for violating these norms.
% Similarly, in Graffiti, while you could spider public channels like a search engine
% to find content about a person, revealing that you've done such a thing
% would be shameful.

% IRC or Slack channels and Matrix rooms are similarly bidirectional and persistent
% but are not global, obscure, ownerless, or crosspostable.
% The web is global, obscure, and persistent
% but not bidirectional, ownerless, or crosspostable.

% ((make this into a table and also include publish/subscribe, e.g. webSub))

% Note that while we have obscurity, there are many channels where you implicitly know
% the string.
% Access control and reifiecation can be combined with channels to create private
% or moderated channels.
% Fortunately, we have access control.
% However this is still not \emph{enough} and channels provide forms of
% social access control in public spaces.

% The fact that channels are ownerless seems ripe for harm, but it is necessary for
% for moderation policies to be ``autonomously extensible'' as per requirement
% \ref{requirements:autonomous-extensibility}.
% This ownerless property is where the system gets its namesake.

% Moderation and other cases of automous extensibility are resolved by our
% notion of ``Interaction Relativity,'' \ref{concepts:interaction-relativity}.

\subsubsection{Other Channel Meanings}

We have seen how channels can represent \emph{people} and \emph{posts}
which cover the contextual needs of most microblogging applications,
but channels can also represent media, topics, websites, and locations,
and more meanings can be freely introduced in an evolving folksonomy~\cite{folksonomy}.
A table of possible channel meanings and the types of applications
those meanings can enable is shown in Table~\ref{concepts:channels-and-applications}.

% Like with object properties, channels represent a \emph{folksonomy}.
% They have no inherent meaning other than the meaning ascribed to them
% by particular application. However, network effects provide incentives for
% application developers to establish shared conventions.
% They will establish meaning by convention and applications have interest
% in establishing the same conventions.

\begin{table}[h]
\small
\caption{Channels Meanings and their Applications}
\label{concepts:channels-and-applications}
\centering
\begin{tabular}{|l|l|}
\hline
\textbf{Channel Meaning} & \textbf{Applications} \\ \hline
External Media & Letterboxd, GoodReads, Genius \\ \hline
User Media & YouTube, Medium, Github \\ \hline
Topics & Reddit, Discord, Slack \\ \hline
Collections & are.na, Pinterest, Spotify playlists \\ \hline
Website & Hypothes.is, Pinboard \\ \hline
Physical Locations & Tinder, Craigslist, Nextdoor \\ \hline
Virtual Locations & Second Life, Minecraft \\ \hline
Events & Partiful, Facebook Events \\ \hline
Products/Services & Amazon reviews, Yelp \\ \hline
Documents & Wikipedia, Google Docs \\ \hline
\end{tabular}
\end{table}

In some cases, an existing standard can be used as a channel
string, like the URL of a website (Hypothes.is),
or the ISBN of a book (GoodReads).
In other cases, the channel string may be user generated,
as with public contexts like SubReddits or a Wikipedia pages,
or randomly generated for ``unlisted'' contexts like
Discord channels, Pinterest boards, or Google documents.
We discuss how continuous contexts, like locations, can be
mapped onto specific channel strings in our Appendix.

% These concatenations may themselves take on new meaning.

% Channels can also represent \emph{application types themselves}.

% For example, to avoid the issue of Linkedin and Tinder
% Channels can also be concatenated.

% Moreover it will not happen by accident --- the spidering
% user ``knew what they were getting into'' and so if they come across information
% the poster
