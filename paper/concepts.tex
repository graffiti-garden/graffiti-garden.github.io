% \section{Design Overview and Concepts}
\section{Design Concepts}

This section outlines the high level concepts that make up the design of Graffiti
according to the requirements we outlined in section \ref{requirements}.
Section~\ref{api} outlines the specific API that is built out of these
concepts and section~\ref{protocols} various protocols that implement the API.

At a high level the design works as follows.
\emph{Actors} (basically users) create \emph{objects} which can represent
social artifacts like posts or activities like "likes".
Objects are associated with a set of \emph{channels} and other users can
discover an object by querying one of its channels. Channels are flexible representation
of context which may be based around an identity, topic, piece of media, location, etc.
Users can only modify their own objects and so to for users to interact with each other's content
(as annotaiton, collaboration, moderation), interactions are reified into indivdual
activies according to a principle we call \emph{interaction relativity}.

\subsection{Actors}

An actor is a persistent and globally unique identifier.
In most cases an actor can be considered synonymous with a user,
however, like an email adress, a user can own multiple actors
or actors may be non-human entities like bots or recommendation services.
The term \emph{actor} was originally used for this purpose
in the ActivityPub standard.

A user can ``log in'' to an actor and then publish as that actor.
It is not possible for one user to impersonate another user's actor.
Importantly to enable full interoperability
(as required by \ref{requirements:adversarial-interop}),
a Graffiti actor is not tied to any particular application so that users can freely move between
different applications.

A Graffiti actor does not include any information other than its
global identifier. For example, an actor does not have a global
name, profile picture, or biography associated with it.
Instead, profiles are defined by seperate profile objects,
allowing applications to freely introduce new profile
metadata or for users to have multiple profiles for different contexts.
For example, a trans person who is not publicly ``out'' may want to
change their name within theirr queer community without
changing their name to the broader public.
Objects and their relation to autonomous extensibility are described next.

For anonymous or meronymous~\cite{meronymous} interactions,
a person can create `throw away' actors.

\subsection{Objects}
\label{concepts:objects}

Objects are the atomic units that encapsulate \emph{all}
of the social data in the system, including
both social artifacts (such as posts and profiles) and social activities (such as likes and follows).
Each Graffiti object include some structured metadata,
such as the actor who created the object,
but an object's value can be \emph{any} valid JSON.
For example, a blog post object may have the value:

\begin{lstlisting}[language=javascript]
{
  title: "My First Post",
  content: "Hello, world!",
  published: 1611972000, // The time in seconds
}
\end{lstlisting}

The ActivityVocabulary~\cite{activity-vocabulary},
whose object model inspired our own\footnote{
    While our use of objects is similar to ActivityPub, one subtle difference is
    that the objects are ``ducktyped'' unlike ActivityPub's use of \texttt{type} properties.
    This allows for slightly more flexibility and interoperability.
},
defines many common properties that are useful for social applications including
the ones above.
However, application developers can freely introduce new properties as necessary
as part of an evolving \emph{folksonomy}~\cite{folksonomy}.
We leverage the fact that JSON can be \emph{self-describing}
--- a message can contain structured data with arbitrary properties and
values that are irrelevant and opaque to the delivery infrastructure but
meaningful to specific clients.
For example, an application for sharing recipes might introduce a new
\texttt{ingredients} property.

Because social applications
benefit from network effects, we expect that developers will make use of
existing properties where possible, allowing them them to benefit from pre-existing data
made in other applications, but will introduce new properties when necessary.
The result will inevitably be messy and inconsistent but, as with the evolution of
particular hashtags, we expect there will be an effective balance of creativity and convergence.
% We also provide tooling that allows users to filter data and provide human-readable
% fallbacks for data they can't parse.

Graffiti objects are also globally addressable which allows them to be
pointed at and linked to.
% Like URLs, these addresses start with scheme which prevents collision
% between different protocols.
For example, a post might have the URI \texttt{graffiti:remote:pod.graffiti.garden/123},
enabling other users to ``like'' that post by creating a new object with the value:

\begin{lstlisting}[language=javascript]
{
  activity: "like",
  target: "https://graffiti.garden/123",
}
\end{lstlisting}

This like activity is an example of \emph{reification},
a concept introduced by the semantic web community
which permits actions on data to be reprepresented as data themselves.
Rather than a post object being an object that has a ``like'' method,
the ``like'' here is a separate object that points to the post object.
Reification allows for new interaction mechanisms to be introduced as easily as new properties,
without changing the underlying system. In other systems only some interactions are reified,
but in Graffiti all interactions, including moderation actions, are reified,
a concept we call \emph{interaction relativity} which we discuss in the next section.


Finally, objects are mutable, meaning that they can be changed or deleted.
This is necessary for the right to be forgotten requirement.
The question of \emph{who} can mutate the objects is determined
by interaction relativity and again discussed in the next section.

In summary Graffiti objects are:

\begin{enumerate}
\item
Flexible: Objects can take on any value according to a self-describing folksonomy.
\item
Globally-Addressable: Every object has a globally unique URI that can be used to fetch it.
\item
Mutable: Objects can be changed or deleted.
% we add the two following contraints on objects
% \item
% Proprietary: Only the creator of an object can modify or delete it.
% \item
% Permissioned: Objects can have an allow list of users who can view them.
% \item
% Ducktyped: Objects have no type.
\end{enumerate}

% Mutable, proprietary, and permissioned are necessary for the right to be forgotten.
% \item
% Objects can be associated with one or more channels.
% \item
% You can either get an object by one of its channels or by its URI.

\subsection{Interaction Relativity}
\label{concepts:interaction-relativity}

We define interaction relativity to mean that ``interaction between two individuals only
exists relative to an application.''
For example if an application is built to understand the like objects,
it might display a counter of all likes.
Alternatively, a user may have added a \texttt{disableLikes} property to
the liked post which, if an application is built to understand it,
may prevent the application from displaying the like counter.
There is no ground truth as to whether the like actually occurred,
only relative interpretations by different applications.
% Whether a user has liked another users post is relative to which
% application they use.

The introduction and interpretation of new properties and objects
that clarify whether an interaction has, or has not, occured
is entirely up to the application developers themselves.
This allows developers to freely introduce new interactions
as required by \ref{requirement:autonomous-extensibility} without
breaking interoperability as required by \ref{requirement:adversarial-interop}.

This becomes unusual when considering moderation.
For example, day that a Moderator wants to delete another user's post.


This helps with that paper that eric gilbert sent. Moderation as data.
For example say that Alice wants to delete a post that Bob made.
For moderation this is like composable moderation. But also more.
For example, one application could have a single fixed moderator,
another could allow users to choose which moderators they would like filter their content
like [Bluesky's stackable moderation](https://bsky.social/about/blog/03-12-2024-stackable-moderation),
and another could implement a fully democratic system like [PolicyKit](https://policykit.org/).
Each of these applications is one interpretation of the underlying refieid user interactions and
users can freely switch between them.

Having flexible object values that we described in section~\ref{concepts:objects} makes reifiecation \emph{possible} but
but interaction relativity requires that not some but \emph{all} interactions
are relative, therefore reification must be \emph{necessary}
for interaction to take place.
For example, it should not be possible for a user to \emph{actually}
delete another users post because that prevents developers from
creating applications where the delete did not occur, violating the
``adversarial'' part of ``adversarial interoperability''
(requirement~\ref{requirement:adversarial-interop}).
As such, we ensure that:
\begin{enumerate}
\item
Objects are \emph{proprietary}: only the creator of an object can modify or delete it.
\end{enumerate}
Conversly, users should not be limited from publishing new objects
which may include new minteraction mechanisms.
Objects are published in channels, a concept we discuss in great detail
in the following section~\ref{concepts:channels}, and therefore
\begin{enumerate}
\setcounter{enumi}{1}
\item
Channels are \emph{ownerless}: no one can prevent others from putting objects in a particular channel.
\end{enumerate}
This ensures that users can always create actions.
that effect

Different applications can have different interpretations of the
underlying state of the system based on s
allowing them to flexibly add more features
This aims to resolve the tension between our requirement for interoperability
and our reqirement for


Another example is collaboration. Say one user would like to modify
Say a user would like to edit another user's post.
a document but another would like to keep it the same. CRDTs.


\subsubsection{Disabled Replies}

Disabling replies a feature of Youtube and Instagram and although not
implemented, one of the most requested features on Mastodon\footnote{
    Mastodon's ``Disabl replies'' feature request has been actively discussed
    for six years
    \url{https://github.com/mastodon/mastodon/issues/8565}
}.
According to our system the replies would not really be disabled
they would just be hidden in application that are built to understand
the disabledReplies property.
Is it possible the \emph{existence} of such replies even if they could
not be seen is harmful?

Unfortunately, if someone is trying to harass

What if though, the person who disabled replies was spreading misinformation?
Wouldn't it be nice for other users to be able to

Users could use disabled replies as a `mic drop'.

Alternatiely,
people from the particular crowd start
@-mentioning Person A in their toots, attacking them,
but disabling replies on these toots, making it harder
for the person to respond.
https://github.com/mastodon/mastodon/issues/8565#issuecomment-417932002

It also means a person cannot be shamed for something they really
want to do.

What about misinformation? There should be both ways for authors to
mark that they want replies disabled or to mark ``approve'' certain comments
and a way for communities to
mark comments as particularly

\subsubsection{Illegal Content}

If moderation is reified and not actually deleting content,
what can be done about universally repugant content such as CSAM?
Here we come up against issues of \emph{trust} which we
have seperated into the \emph{protocol} layer rather than
the API layer.
End-to-end encrypted systems, like Signal, make it impossible
for users to be surveilled but also impossible to detect illegal content.
In centralized unencrypted systems it may be possible to do both, but
users must trust the system to act in their best interest.
% In certain decentralized systems, like BitTorrent, IPFS, and Blockchain
% it is possible to detect illegal content but impossible to delete it.

At the moment this appears to be a trolley problem:
if users can have complete trust in the system to act in their
best interest the system will also be acting in the best interest
of bad actors.
perhaps technological advancements will make it possible to create
a system that users can trust to act in their best interest,
while also keep out users doing truly harmful things.
We demonstrate that multiple protocols can implement the
Graffiti API and leave it up to users to decide which they want
and leave it up to other researchers with more knowledge of
and for future work to find better soltions.

% This works for moderation of legal content but there is the question of
% what about content that doesn't want to exist anywhere, such as CSAM.
% Mechanisms to delete content are dependent on the underlying implementation of the
% system. In some systems, such as end-to-end encrypted systems like Signal,
% and WhatsApp, it is impossible detect such content. In distributed systems
% like BitTorrent, IPFS, or Blockchain technologies it is impossible to delete content.

Interaction relativy also allows applications to introduce new sorts of interactions
without having to coordinate with all the other existing applications,
keeping the ecosystem flexible and interoperable.
For example, an application could [add a "Trust" button to posts](https://social.cs.washington.edu/pub_details.html?id=trustnet)
and use it assess the truthfulness of posts made on applications across Graffiti.
New sorts of interactions like these can be smoothly absorbed by the broader ecosystem
as a [folksonomy](https://en.wikipedia.org/wiki/Folksonomy).

\subsection{Channels}
\label{concepts:channels}

Channels are our solution to mitigating context collapse, as required by
\ref{requirements:context-differentiation}, when still being
interoperable as required by~\ref{requirements:adversarial-interop}.
Like a television channel, users can publish content (in this case
objects, see~\ref{concepts:objects}) to a particular channel
and user watching that channel will receive it.
Unlike a television channel, Graffiti channels are:
\begin{enumerate}
\item
Bidirectional: Users can both read from and write to channels.
\item
Persistent: When a user publishes an object to a channel, other users can
continue to retrieve the object from the channel until it gets deleted.
\item
Global: The channel namespace is global with every string (up to a certain length)
corresponding to exactly one channel.
\item
Obscure: It is not possible to read from or write to a channel without knowing
its string.
\item
Crosspostable: An object can be associated with more than one channel (or no channel at all).
\end{enumerate}

Additionally, to enable interaction relativity, we mentioned preemptively that channels are ownerless.
Combined with obscurity,

\begin{enumerate}
\setcounter{enumi}{5}
\item
Ownerless: No user can prevent any other user from reading from or writing to a channel
\emph{that they know about}.
\end{enumerate}

IRC or Slack channels and Matrix rooms are similarly bidirectional and persistent
but are not global, obscure, ownerless, or crosspostable.
The World Wide Web is global, obscure, and persistent
but not bidirectional, ownerless, or crosspostable.

% The fact that channels are ownerless seems ripe for harm, but it is necessary for
% for moderation policies to be ``autonomously extensible'' as per requirement
% \ref{requirements:autonomous-extensibility}.
% This ownerless property is where the system gets its namesake.

% Moderation and other cases of automous extensibility are resolved by our
% notion of ``Interaction Relativity,'' \ref{concepts:interaction-relativity}.

\subsubsection{Replies}

\begin{table}[htbp]
    \label{concepts:channel-replies}
    \centering
    \caption{Channels and Replies}
    \begin{tabular}{c|c|c|c|}
    \multicolumn{2}{c}{}
    & \multicolumn{2}{c}{\textbf{Posted to reply context?}}
    \multicolumn{2}{c|}{}
    & \textbf{Yes} & \textbf{No}
    \textbf{Posted to}
    & \textbf{Yes}
    & Twitter/X Reply & Quote Tweet
    \textbf{self context?}
    \\
    & \textbf{No}
    & Instagram Comment & N/A
    \end{tabular}
\end{table}

To understand how channels can work to prevent context collapse, consider the example of \emph{replies}.
Many social applications allow you to reply to posts but there are suprisingly subtle
contextual differences between different reply designs.
For example, currently on Instagram
if you reply to a post that reply can only be seen by other
people also viewing that particular post.
However, prior to 2019 Instagram had a Following tab where a user could see
the comments that the people they followed had posted across the application.
The Following tab led to a priest outing himself as gay among numerous other
examples of context collapse [CITE buzzfeed article].
On Twitter/X it is still possible to see all of a users' replies by going
to their Replies tab.

With channels, it is possible to implement replies that allow users to
prevent context collapse with the X, while still Y.

To make an Instagram-style reply, that reply should be posted to the channel
containing the original post's URI.
Applications displaying that post will know to query this particular channel
but they will not
If we also place the comment in the channel represented by the commenter's URI (their
{@link GraffitiObjectBase.actor | `actor` URI}), then people viewing the commenter's profile
will also see the comment, giving it more visibility, like a reply on Twitter.
If we *only* place the comment in the channel represented by the commenter's URI, then
it becomes like a quote tweet ([prior to 2020](https://x.com/Support/status/1300555325750292480)),
where the comment is only visible to the commenter's followers but not the audience
of the original post.

Importantly context is preserved across interoperation. If a user posts their reply in
an Instagram-like app they should expect that their reply will not be leaked even if
a user is using Twitter.

\subsubsection{Other Uses of Channels}

We've seen how channels can represent \emph{people} and \emph{posts}
which covers most microblogging uses of channels, but channels can also represent,

\begin{enumerate}
\item
External Media: for applications like Letterboxd, GoodReads, Genius, last.fm
\item
User-Created Media: for applications like Soundcloud, Youtube, Medium
\item
User-Generated Topics: for applications like Reddit, Discord, Slack, IRC
\item
User-Generated Collections: for applications like are.na, Pinterest, Spotify playlists
\item
Website: for applications like Hypothes.is, Pinboard, Delicious
\item
Physical Locations: for location-based apps like Tinder, Nextdoor, Craigslist, and YikYak.
\item
Virtual Locations: for applications like VRChat, Second Life, Minecraft
\item
Events: for applications like Meetup, Partiful, Facebook Events
\item
Products and Services: Amazon reviews, Yelp,
\item
Collaborative documents: Wikipedia, google Docs
\end{enumerate}

Channels can also represent \emph{application types themselves}.
For example, to avoid the issue of Linkedin and Tinder.
Channels can also be concatenated.
For example, `dating+location' would prevent other people from seeing th post.
Concatenation may also happen in natural language.
For example, on Reddit when one wants to post programming-related memes they post to \texttt{r/programmerhumor} rather than \texttt{r/programming+funny}.
These concatenations may themselves take on new meaning.

Locations are interesting because they are a continuous space.
Something like what-three-words, or a "resolution pyramid".
Locations are also an example of when users without any prior interaction
post to the same channel.

Websites make it possible for everyone to annotate the web.
In fact, Graffiti's namesake comes from an article
describing an early web annotation system, Third Voice,
as Graffiti on the web.
The authors meant it as a slur, but we believe graffiti
can be a beautiful art form and a powerful tool for social change.

\subsubsection{Notes}

One particular benefit of the channel concept over near misses, is that channels more naturally allow for BCCing.
While many of the channels a user posts to are presumably related to properties of the data itself, it is also possible to post data to multiple channels without revealing those channels to each other.
For example, a user may be part of several different contexts invite them to a party with a universal thread, but doesn't want to inundate one with another by including them directly.

\subsubsection{Comparison to Other models}

While channels are a relatively simple, they are subtly differences from other established concepts:

The channel model is clearly different from the Firehose Model used by BlueSky
which.
The firehose model dumps all user data into one public database,
which doesn't allow for the carving out of different contexts that we did in our comment
example above. In the firehose model a comment will always be visible to *both* the original post's audience and
the commenter's followers.
It also has subtle differences.
There are several other contextual models.

\paragraph{Actor}

We use actors to describe identities in the system,
but ActivityPub uses the actor model to describe the system itself.
As mentioned the actor model used by ActivityPub is a fusion of
direct messaging (like Email) and broadcasting like RSS.
Having channels

Compared to the Actor Model it removes the need for Server-Side effects
The actor model is a fusion of direct messaging (like Email) and broadcasting
(like RSS) and works well for follow-based communication but struggles
to pass information via other rendez-vous.
In the actor model, even something as simple as comments can be
[very tricky and require server "side effects"](https://seb.jambor.dev/posts/understanding-activitypub-part-3-the-state-of-mastodon/).

\paragraph{Bidirectional Links}

On the World Wide Web, hyperlinks point in one direction from one website to another.
However earlier hypermedia systems, like Xanadu and Dexter, included \emph{bidirectional} links:
from one document you could see all the other documents that link to it.
Bidirecitonal links are now currently possible in personal information management systems like Roam and Notion.

Channels can be seen as a way to \emph{selectively} create bidirectional links.
A reply object will have pointers to the object it is replying to and the actor who created the reply.
Adding the object's URI or the actor's URI as channels selectively adds backlinks to those entities
making it possible to go \emph{from} the original post \emph{to} the reply (in the case of Instagram-like comments)
or \emph{from} the actor \emph{to} the reply (in the case of Twitter/X-like comments).
Of course, channels also make it possible to create "links" from non-digital artifacts like topics or locations.

\paragraph{Object Capabilities}

Object capability security (OCaps) is a security model where rights to perform an action
can be transferred to another user by giving that user a reference to that action.
Channels can be thought of as a capability mechanism where
capability read or write to a particular channel
can be transferred to another user by telling them the channel's string.

However Graffiti isn't a purely capability-based system because it
is also possible to add an "allow" list to each object.
For example a group chat could use a private string as a channel
and if that string is sufficiently long and random so as not to be guessed
the conversation would remain private.
However this makes it possible for any member of the group to share (or accidentally leak) the
channel and let anyone else in.
Allow lists provide more long term privacy.

Allow lists are managed by applications.
If allow lists are especially enourmous then is it really public?

\subsubsection{Social Access Control}

In some sense, channels provide a sort of "social access control" by forming
expectations about the audiences of different online spaces.
As a real world analogy, oftentimes support groups, such as alcoholics
anonymous, are open to the public but people in those spaces feel comfortable sharing intimate details
because they have expectations about the other people attending.
If someone malicious went to support groups just to spread people's secrets,
they would be shamed for violating these norms.
Similarly, in Graffiti, while you could spider public channels like a search engine
to find content about a person, revealing that you've done such a thing
would be shameful.

Still, social access control is not perfect and so in situations where privacy is important,
objects can also be given
an {@link GraffitiObjectBase.allowed | `allowed`} list.
For example, to send someone a direct message you should put an object representing
that message in the channel that represents them (their {@link GraffitiObjectBase.actor | `actor` URI}),
so they can find it, *and* set the `allowed` field to only include the recipient,
so only they can read it.

\subsubsection{Near Misses}

Compared to near misses, the channel concept allows data recipients to only make queries on the \emph{union} of attributes --- a query can be for posts by Alice \emph{or} in reply to the target post but not for posts by Alice \emph{and} in reply to the target post. The recipient is left to do further filtering and in our implementation section we discuss the consequences this has on performance and client-side only design.

It is possible to \emph{intersect} channels by concatenating them.
For example, if we only want a message to be seen by people interested in dating at a particular
location the channel would be
The order of the concatentation is ambiguous and there is the potential.

These union queries also limit the options that a poster has to specify where they want data to appear. However, we believe that in most cases where a user wants content to only be available in the intersection of two channels, it is appropriate to create a new channel through either automated or natural language concatenation.

One possible generalization of `Channels' is near misses.
The concept is hard to develop and the scope in which such logical interections
might be useful is limited. However for completness we thought it is worth mentioning.

BCCs don't work with near misses.

One way to operationalize public contexts is similar to how people set boundaries in natural language.
When asked a general question, like ``How are you?'', many people respond with a general answer, ``Good, and you?''
When asked a more specific question, like ``What did you think of the game last night?'', many people respond with a more specific answer.

Asking a more specific question can be a shibolleth
signals the querier's interests
which can give the responder confidence they are not oversharing. A more specific question may also signal whether the querier is part of some in-group, allowing the responder to disclose information that may technically be public but not readily shared (``Are you a friend of Dorothy's?'').

This natural language analogy is a far cry from how database queries work.
For example, in the Nostr infrastructure that allows users to make general queries on each others social data, a simple query for a users ID returns everything they've ever posted. This is the equivalent of asking someone ``How are you?'' and then sitting through their entire life history.

What we are suggesting is that somehow when a user posts data, they should also be able to outline \emph{how specific} a query must be in order to retrieve that data.
This way, while the data is still public, it won't be found by users not asking the ``right question.''
Again, this is not about absolute privacy, which can be handled via standard access control, but about giving a content creator situational privacy about the audience that will consume their data.

A first attempt would be for users to mask out fields they don't want queried.
For example, suppose a user \texttt{alice@example.com} wants to reply to a post that can be identified with the URL \texttt{example.com/posts/12345}.
Using the Activity Vocabulary, we can represent Alice's reply as follows:

\begin{verbatim}
{
   type: 'Note',
   content: "Thank you for a great post!",
   inReplyTo: example.com/posts/12345,
   actor: alice@example.com
}
\end{verbatim}

Alice masks out all fields except for \texttt{inReplyTo} and gives the redacted post to an indexer.
When someone queries for \texttt{\{ actor: alice@example.com \}} they won't be able to find the data, but when they query for \texttt{\{ inReplyTo: example.com/posts/12345 \}} they will.
Unfortunately, if a friend of Alice's is only interested in her replies and queries for the appropriate actor \emph{and} reply fields, they won't see Alice's reply.

One way to solve this generally and consistently by borrowing the concept of \emph{near misses} from classical artificial intelligence.
When training a classifier, a near miss is something that is very close to being an example of a class of the concept without \emph{actually} being an example of that class.
We can apply this to queries: to retrieve a piece of data a query must match the data, but must \emph{not} match the near misses.

In our example, Alice's near miss should be examply the same as her post data, only
the \texttt{inReplyTo} field should be changed to a random URL.
A query for Alice's actor will match both will match both the reply \emph{and} its near miss, and so the reply \emph{will not} be returned.
A query for posts \texttt{inReplyTo} the target post will match the reply but \emph{not} its near miss and so the reply \emph{will} be returned.
Importantly, queries for posts that are both by Alice \emph{and} replies to the target post will also match the reply, but not its near miss and so will be returned.

This system is quite general and with multiple near misses it can be used to qualify ``how specific'' a query is with arbitrary logical intersections.
However, we found in initial prototypes that its contrapositive formulation is extremely confusing.
Moreover, aside from some contrived examples, all use cases could be described by a simpler concept: \emph{channels}.
