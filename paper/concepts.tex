\usepackage{xr-hyper}

\section{Design Overview and Concepts}


\subsection{API Over Protocol}

\ref{req:easy}

Readers may have noticed that there was no mention of privacy,
or decentralization in oura document.
These are certainly things we believe in but they can be met in many different ways.
The tech is evolving.

It's a lot like a URL. Browsers can understand multiple URL schemes: http, https, ipfs, ftp, mailto, tel, magnet, etc. Each URL is associated with a very different protocol. But I can type any of them into my browser bar and, if the browser supports that protocol, it will return the right result / open the right application, etc.

\subsection{Channels}

Channels are a way for the creators of social data to express the intended audience of their
data. When a user creates data using the {@link put} method, they
can place their data in one or more channels.
Content consumers using the {@link discover} method will only see data
contained in one of the channels they specify.

While many channels may be public, they partition
the public into different "contexts", mitigating the
phenomenon of [context collapse](https://en.wikipedia.org/wiki/Context_collapse) or the "flattening of multiple audiences."
Any [URI](https://en.wikipedia.org/wiki/Uniform_Resource_Identifier) can be used as a channel, and so channels can represent people,
comment threads, topics, places (real or virtual), pieces of media, and more.

For example, consider a comment on a post. If we place that comment in the channel
represented by the post's URI, then only people viewing the post will know to
look in that channel, giving it visibility akin to a comment on a blog post
or comment on Instagram ([since 2019](https://www.buzzfeednews.com/article/katienotopoulos/instagrams-following-activity-tab-is-going-away)).
If we also place the comment in the channel represented by the commenter's URI (their
{@link GraffitiObjectBase.actor | `actor` URI}), then people viewing the commenter's profile
will also see the comment, giving it more visibility, like a reply on Twitter.
If we *only* place the comment in the channel represented by the commenter's URI, then
it becomes like a quote tweet ([prior to 2020](https://x.com/Support/status/1300555325750292480)),
where the comment is only visible to the commenter's followers but not the audience
of the original post.

The channel model differs from other models of communication such as the
[actor model](https://www.w3.org/TR/activitypub/#Overview) used by ActivityPub,
the protocol underlying Mastodon, or the [firehose model](https://bsky.social/about/blog/5-5-2023-federation-architecture)
used by the AT Protocol, the protocol underlying BlueSky.
The actor model is a fusion of direct messaging (like Email) and broadcasting
(like RSS) and works well for follow-based communication but struggles
to pass information via other rendez-vous.
In the actor model, even something as simple as comments can be
[very tricky and require server "side effects"](https://seb.jambor.dev/posts/understanding-activitypub-part-3-the-state-of-mastodon/).
The firehose model dumps all user data into one public database,
which doesn't allow for the carving out of different contexts that we did in our comment
example above. In the firehose model a comment will always be visible to *both* the original post's audience and
the commenter's followers.

In some sense, channels provide a sort of "social access control" by forming
expectations about the audiences of different online spaces.
As a real world analogy, oftentimes support groups, such as alcoholics
anonymous, are open to the public but people in those spaces feel comfortable sharing intimate details
because they have expectations about the other people attending.
If someone malicious went to support groups just to spread people's secrets,
they would be shamed for violating these norms.
Similarly, in Graffiti, while you could spider public channels like a search engine
to find content about a person, revealing that you've done such a thing
would be shameful.

Still, social access control is not perfect and so in situations where privacy is important,
objects can also be given
an {@link GraffitiObjectBase.allowed | `allowed`} list.
For example, to send someone a direct message you should put an object representing
that message in the channel that represents them (their {@link GraffitiObjectBase.actor | `actor` URI}),
so they can find it, *and* set the `allowed` field to only include the recipient,
so only they can read it.

The channel concept lets users to post data to one or more channels and lets users to fetch all the data associated with a particular channel.
Channels do not need to be created and are not owned by anyone. Instead, a channel can be represented by any URI: a user identifier, a post, a topic, a place, etc.

In our running example, Alice would post her reply to the channel \texttt{inReplyTo:example.com/posts/12345} so that users interested in replies to the target post could find it.
If Alice wanted her post to be Twitter-like, she would also post their data to the channel \texttt{actor:alice@example.com}.
If we complete the square and consider what happens if Alice posts to her actor channel, but not the reply channel, her post becomes like a Quote Tweet, as shown in figure 1.

\begin{table}[htbp]
    \centering
    \caption{Context and Replies}
    \label{tab:reply_dimensions}
    \begin{tabular}{c|c|c|c|}
    \multicolumn{2}{c}{}
    & \multicolumn{2}{c}{\textbf{Posted to reply context?}}
    \\ \cline{3-4}
    \multicolumn{2}{c|}{}
    & \textbf{Yes} & \textbf{No}
    \\ \cline{2-4}
    \textbf{Posted to}
    & \textbf{Yes}
    & Twitter/X Reply & Quote Tweet
    \\ \cline{2-4}
    \textbf{self context?}
    & \textbf{No}
    & Instagram Comment & N/A
    \\ \cline{2-4}
    \end{tabular}
\end{table}

Compared to near misses, the channel concept allows data recipients to only make queries on the \emph{union} of attributes --- a query can be for posts by Alice \emph{or} in reply to the target post but not for posts by Alice \emph{and} in reply to the target post. The recipient is left to do further filtering and in our implementation section we discuss the consequences this has on performance and client-side only design.

These union queries also limit the options that a poster has to specify where they want data to appear. However, we believe that in most cases where a user wants content to only be available in the intersection of two channels, it is appropriate to create a new channel through either automated or natural language concatenation.
For example, on Reddit when one wants to post programming-related memes they post to \texttt{r/programmerhumor} rather than an intersection of \texttt{r/programming} and \texttt{r/funny}.

One particular benefit of the channel concept over near misses, is that channels more naturally allow for BCCing.
While many of the channels a user posts to are presumably related to properties of the data itself, it is also possible to post data to multiple channels without revealing those channels to each other.
For example, a user may be part of several different contexts invite them to a party with a universal thread, but doesn't want to inundate one with another by including them directly.

While channels are a relatively simple, they are subtly differences from other established concepts:


\paragraph{Rooms}

Channels are similar to the concept of \emph{rooms} used by Matrix and also by the teleconfrencing platform, Zoom.
However unlike rooms, which have an owner that can kick people out, channels are public. This is because channels are a way for senders to limit who their recipients are not necessarily for recipients to limit who their senders are. The latter falls under moderation, which we address in the next subsection.

\paragraph{Bidirectional Linking}

% Channels can be thought as a selective form of bidirectional linking
% Cite Xanadu, Dexter, Third Voice, PXSI.

\paragraph{Object Capabilities}

% Object Capabilities are a security model where rights to an perform an action can be transferred to another user by giving that user a reference to that action.
% For example to read
% We rely on standard access control for long term.
% Channels are a form of object capability. They are a \emph{handle} to underlying content: knowing the ``right question to ask.''

\subsubsection{Near Misses}

One way to operationalize public contexts is similar to how people set boundaries in natural language.
When asked a general question, like ``How are you?'', many people respond with a general answer, ``Good, and you?''
When asked a more specific question, like ``What did you think of the game last night?'', many people respond with a more specific answer.

Asking a more specific question signals the querier's interests which can give the responder confidence they are not oversharing. A more specific question may also signal whether the querier is part of some in-group, allowing the responder to disclose information that may technically be public but not readily shared (``Are you a friend of Dorothy's?'').

This natural language analogy is a far cry from how database queries work.
For example, in the Nostr infrastructure that allows users to make general queries on each others social data, a simple query for a users ID returns everything they've ever posted. This is the equivalent of asking someone ``How are you?'' and then sitting through their entire life history.

What we are suggesting is that somehow when a user posts data, they should also be able to outline \emph{how specific} a query must be in order to retrieve that data.
This way, while the data is still public, it won't be found by users not asking the ``right question.''
Again, this is not about absolute privacy, which can be handled via standard access control, but about giving a content creator situational privacy about the audience that will consume their data.

A first attempt would be for users to mask out fields they don't want queried.
For example, suppose a user \texttt{alice@example.com} wants to reply to a post that can be identified with the URL \texttt{example.com/posts/12345}.
Using the Activity Vocabulary, we can represent Alice's reply as follows:

\begin{verbatim}
{
   type: 'Note',
   content: "Thank you for a great post!",
   inReplyTo: example.com/posts/12345,
   actor: alice@example.com
}
\end{verbatim}

Alice masks out all fields except for \texttt{inReplyTo} and gives the redacted post to an indexer.
When someone queries for \texttt{\{ actor: alice@example.com \}} they won't be able to find the data, but when they query for \texttt{\{ inReplyTo: example.com/posts/12345 \}} they will.
Unfortunately, if a friend of Alice's is only interested in her replies and queries for the appropriate actor \emph{and} reply fields, they won't see Alice's reply.

One way to solve this generally and consistently by borrowing the concept of \emph{near misses} from classical artificial intelligence.
When training a classifier, a near miss is something that is very close to being an example of a class of the concept without \emph{actually} being an example of that class.
We can apply this to queries: to retrieve a piece of data a query must match the data, but must \emph{not} match the near misses.

In our example, Alice's near miss should be examply the same as her post data, only
the \texttt{inReplyTo} field should be changed to a random URL.
A query for Alice's actor will match both will match both the reply \emph{and} its near miss, and so the reply \emph{will not} be returned.
A query for posts \texttt{inReplyTo} the target post will match the reply but \emph{not} its near miss and so the reply \emph{will} be returned.
Importantly, queries for posts that are both by Alice \emph{and} replies to the target post will also match the reply, but not its near miss and so will be returned.

This system is quite general and with multiple near misses it can be used to qualify ``how specific'' a query is with arbitrary logical intersections.
However, we found in initial prototypes that its contrapositive formulation is extremely confusing.
Moreover, aside from some contrived examples, all use cases could be described by a simpler concept: \emph{channels}.

\subsection{Interaction Relativity}

Interaction relativity posits that "interaction between two individuals only
exists relative to an observer," or equivalently, all interaction is [reified](https://en.wikipedia.org/wiki/Reification_(computer_science)).
For example, if one user creates a post and another user wants to "like" that post,
their like is not modifying the original post, it is simply another data object that points
to the post being liked, via its {@link locationToUri | URI}.

```json
{
  activity: 'like',
  target: 'uri-of-the-post-i-like',
  actor: 'my-user-id'
}
```

In Graffiti, all interactions including *moderation* and *collaboration* are relative.
This means that applications can freely choose which interactions
they want to express to their users and how.
For example, one application could have a single fixed moderator,
another could allow users to choose which moderators they would like filter their content
like [Bluesky's stackable moderation](https://bsky.social/about/blog/03-12-2024-stackable-moderation),
and another could implement a fully democratic system like [PolicyKit](https://policykit.org/).
Each of these applications is one interpretation of the underlying refieid user interactions and
users can freely switch between them.

Interaction relativy also allows applications to introduce new sorts of interactions
without having to coordinate with all the other existing applications,
keeping the ecosystem flexible and interoperable.
For example, an application could [add a "Trust" button to posts](https://social.cs.washington.edu/pub_details.html?id=trustnet)
and use it assess the truthfulness of posts made on applications across Graffiti.
New sorts of interactions like these can be smoothly absorbed by the broader ecosystem
as a [folksonomy](https://en.wikipedia.org/wiki/Folksonomy).

Interactivy relativity is realized in Graffiti through two design decisions:
1. The creators of objects can only modify their own objects. It is important for
   users to be able to change and delete their own content to respect their
   [right to be forgotten](https://en.wikipedia.org/wiki/Right_to_be_forgotten),
   but beyond self-correction and self-censorship all other interaction is reified.
   Many interactions can be reified via pointers, as in the "like" example above, and collaborative
   edits can be refieid via [CRDTs](https://en.wikipedia.org/wiki/Conflict-free_replicated_data_type).
2. No one owns channels. Unlike IRC/Slack channels or [Matrix rooms](https://matrix.org/docs/matrix-concepts/rooms_and_events/),
   anyone can post to any channel, so long as they know the URI of that channel.
   It is up to applications to hide content from channels either according to manual
   filters or in response to user action.
   For example, a user may create a post with the flag `disableReplies`.
   Applications could then filter out any content from the replies channel
   that the original poster has not specifically approved.
