\section{Conclusion and Future Work}

\hl{%
Graffiti seeks to enable a ``world where many worlds fit''~{\cite{escobarpluriverse}},
comprised of diverse social spaces that are each simultaneously \emph{personal}
and \emph{interoperable} with each other.
We introduce the concepts \emph{total reification} and \emph{channels}
to resolve the tension between these competing goals
of minimal constraint and interconnectedness.
}%

\hl{%
In such an unconstrained environment,
we hypothesize that the social pressures that currently work
to consolidate people on a handful of massive platforms
will similarly nudge users of Graffiti toward interoperability.
Specifically, we predict that natural \emph{folksonomies}~{\cite{folksonomy}} will
develop around the usage of channels and object schemas, much as they have with hashtags.
It remains to be seen whether this hypothesis will hold in wide deployment,
or whether there will be instances where too much freedom leads to
``expression breakdowns''~{\cite{expressionbreakdowns}}
that require additional layers of standardization.
}%

\hl{%
Our work illustrates that by placing an \emph{API}
at the right level of the stack, we can separate out the distinct concerns
of building a usable application development environment
from concerns of building a secure, scalable, and decentralized infrastructure.
We demonstrate this by developing a Vue.js plugin and a series
of applications \emph{above} the API, independently from the
two decentralized implementations we present \emph{below}.
However, this hourglass design assumes the API in the middle is static,
which begs the question: is our API sufficient?
}%

\hl{%
Our case studies span the Form-From taxonomy of social media~{\cite{formfrom}},
by demonstrating commons (Wikiffiti), spaces (Parallax, The Glue Factory), and networks (Glitter),
as well as both threaded (Glitter) and unthreaded (Parallax) interactions.
Channels and allowed lists respectively capture the two major
forms of context management available in social applications:
communication toward both abstract and targeted imagined audiences~{\cite{imaginedaudience}},
and communication toward a specific audience through ``visibility control''~{\cite{visibilitycontrol}} affordances.
Graffiti interactions can be ephemeral or persistent, real-time or asynchronous,
and moderated with practically any rule set.
}%

\hl{%
However, while the Graffiti API theoretically provides the necessary
affordances for retrieving social data of interest,
there are cases where our client-side
approach makes processing that data infeasible.
For example, it is impossible for any single consumer device to
perform the processing required to create TikTok's For You feed.
For these cases, future work should develop
a parallel API for offloading such
computation to third-party services.
Similar care should be taken
in the design of that API to prevent lock-in.
}%

\hl{%
Lastly, we framed the motivation behind Graffiti in terms of
\emph{individual} and \emph{community}-centric concerns,
specifically, the desire to personalize social spaces while remaining connected to others.
Our work does not directly address \emph{society}-level concerns,
such as disinformation and polarization.
Our hope, in line with predictions made by~{\cite{threeleggedstool}}, is that by democratizing
the design of communication environments---as Graffiti works to do---we
will see the bottom-up emergence of healthier social interactions within those environments,
at both community and society scales.
We look forward to continuing to work toward that vision.
}%


% Our optimistic hope is that by democratizing the design of
% communication infrastrucutre, including moderation, there are downstream effects of creating
% a healthier ecosystem more broadly.

% Call to action.
% More accessible

% We call to action work on making this more accessible

% communication infrastructure,
% this has downstream effects for creating a healthier communication ecosystem more broadly.
% Future work should

% democratization of the design of spaces,
% leads naturally to a healthier communication ecosystem


% We hypothesize that the democratization that Graffiti enables
% makes it a sufficiently more \emph{neutral} interaction environment than
% profit-driven platforms.
% Our optimistic hope such a truly democratic environment,
% will foster not only the creation of good apps but also of good ideas
% good ideas will prevail.
% However, to realize this vision, there is more work to be done making
% the capabilities that Graffiti provides even more accessible.
% However, to realize this, there is work to be done in building even more
% accessible higher-level tools on top of Graffiti for community care and accountability.





% We have attempted to make the API minimal while covering lots of use cases,
% however there are areas where it is likely lacking, namely in areas that require
% analyzing \emph{massive} portions of the networks.
% Future work should attempt to create third-party aggregation services,
% while also working to make sure that users do not get locked into these services.

% Another important question is whether our implementation is \emph{efficient}
% enough to support real-world use cases.
% For example, a decentralized system must consider how to handle CSAM material, and we explore that issue there.
%DK also mention that graffiti's issues with CSAM are no worse than whatsapp's



%Additionally, total reification also makes tasks
%such as tallying up the millions of likes on
%one of Taylor Swift's posts or compiling all the edits on
%Taylor's Wikipedia page a rather arduous challenge.
%For all of these situations, there must be standards for applications
%to offload massive and repetitive computation to third party services,
%although there must also be ways to freely switch between these
%services so they do not begin to lock the network into a particular design.


% Finally, what will the social application
% landscape look like when it design is driven by user needs
% rather than the profits motives of a few corporate platforms or
% the constraints of existing protocols?
% Will a new feature as canonical as the like button or the feed emerge?
% Or will the ecosystem be marked by its lack of canon?
% What would a \emph{tangible} social application feel like?



% as making them becomes more accessible
% and they don't need unanimous support to be

% when not purely directed by broad appeal
% and lock-in potential.
% What social applications are really everything

% adoption.


% and fostering community self-determination
% can help to create accountability and aid networks that work better
% than paternalistic policing to combat harms like CSAM or

% In terms of Graffiti's scope, applications like Tik

% The Graffiti API is curre


% Graffiti demonstrates
% % that it is possible to build a wide variety
% % of social interoperating applications on top of a set of primitives.
% % While this paper has described a system that technically works to satisfy our requirements, we have left out some of the social questions: What new potential does this have for harm? Or Good?

% Additionally, while the API is capable of handling many types of applications, how is performance effected aggregating all the likes on one of Taylor Swift's posts, compiling the edits of a popular wikipedia page, or computing TikTok's For You feed. For these, some third party service is necessary, but ideally the user would be able to opt-in or out of these services as freely as with other applications.


% There are questions about its harm. For content like
% CSAM, Graffiti does not make it any more possible
% to host than hosting it on other decentralized systems like the website, or a Mastodon instance,
% or end-to-end encrypted system.
% However, just as how addiction is better handled with
% community care than policing, the healthier communities that Graffiti has the potential to create may provide a pathway
% to get rid of such content without force.
%
% So much of our lives are online

% \begin{enumerate}
%     \item
%     Recommendation engines and Taylor Swift
%     \item
%     Large media / streaming:
% A major limitation of the API as it currently stands
% is its support for large media and streaming.
% Currently, in applications like The Gloof, images
% are stored externally and linked in. In other applications
% we did not write about images are encoded with Base64.
% \item
% Social questions: norms? harms? a world without hierarchical moderation / paternalism? filter bubbles? polarization? CSAM?
% \end{enumerate}


% We balance the tensions between creating a malleable environment
% and an interconnected environment with concepts like
% total reification and channels.
% Our hope is that this
% minimally constrained environment
% for building social media much as the
% the web did for publishing static media~\cite{weavingtheweb}.
% Will everyone use their own app will there be patterns
% connected digital spaces like MySpace or will people tend
% to use aggregators like RSS reads that avoid the need to
