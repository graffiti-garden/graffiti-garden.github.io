\section{Conclusion and Future Work}

Graffiti seeks to enable a ``world where many worlds fit''~{\cite{escobarpluriverse}},
comprised of diverse social spaces that are each simultaneously \emph{personal}
and \emph{interoperable} with each other.
We introduce the concepts \emph{total reification} and \emph{channels}
to resolve the tension between these competing goals
of minimal constraint and interconnectedness.

In such an unconstrained environment,
our running hypothesis is
that the social pressures that currently work
to consolidate people on a handful of massive platforms
will similarly nudge users of Graffiti toward interoperability.
Specifically, we predict that natural \emph{folksonomies}~{\cite{folksonomy}} will
develop around the usage of channels and object schemas, much as they have with hashtags.
It remains to be seen whether this hypothesis will hold in wide deployment,
or whether there will be instances where too much freedom leads to
``expression breakdowns''~{\cite{expressionbreakdowns}}
that require additional layers of standardization.

Our work illustrates that by placing an \emph{API}
at the right level of the stack, we can separate out the distinct concerns
of building a usable application development environment
from concerns of building a secure, scalable, and decentralized infrastructure.
We demonstrate this by developing a Vue plugin and a series
of applications \emph{above} the API, independently from the
two decentralized implementations we present \emph{below}.
However, this hourglass design assumes the API in the middle is static,
which begs the question: is our API sufficient?

Our case studies span the Form-From taxonomy of social media~{\cite{formfrom}},
by demonstrating commons (Wikiffiti), spaces (Parallax, The Glue Factory), and networks (Glitter),
as well as both threaded (Glitter) and unthreaded (Parallax) interactions.
Channels and allowed lists respectively capture the two major
forms of context management available in social applications:
communication toward both abstract and targeted imagined audiences~{\cite{imaginedaudience}},
and communication toward a specific audience through ``visibility control''~{\cite{visibilitycontrol}} affordances.
Graffiti interactions can be ephemeral or persistent, real-time or asynchronous,
and moderated with practically any rule set.

However, while the Graffiti API theoretically provides the necessary
affordances for retrieving social data of interest,
there are cases where our client-side approach inevitably breaks down.
For example, it is unnecessarily difficult for a consumer device to
aggregate millions of reified likes on a celebrity's post into a single count, and
downright impossible for such a device to
process millions of videos to construct an analog of TikTok's For You feed.
For these cases, future work should develop
a parallel API for offloading such
computation to third-party services.
Similar care should be taken
in the design of that API to prevent lock-in.

Lastly, we framed the motivation behind Graffiti in terms of
\emph{individual} and \emph{community}-centric concerns,
specifically, the desire to personalize social spaces while remaining connected to others.
Our work does not directly address \emph{society}-level concerns,
such as disinformation and polarization.
Our hope, in line with predictions made by~{\cite{threeleggedstool}}, is that by democratizing
the design of communication environments---as Graffiti works to do---we
will see the bottom-up emergence of healthier social interactions within those environments,
at both community and society scales.
We look forward to continuing to work toward that vision.
