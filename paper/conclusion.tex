\section{Conclusion}

Graffiti seeks to enable a ``world where many worlds fit''~\cite{escobarpluriverse}.
So far we have focused on the positive examples of what
worlds are possible, of which there are many,
but what of the harms?
The gravity of the subject requires a full treatment,
which can be found in our companion manuscript~\cite{companion}.
Briefly, we argue that despite Graffiti's personalization, the fuzzy boundaries between Graffiti applications can work to \emph{reduce} filter bubbles, and despite Graffiti's lack of a central authority, the evolving networks of care and accountability that can be built on it will be \emph{more effective} than paternalistic policing in combating harms like CSAM.
% For example, a decentralized system must consider how to handle CSAM material, and we explore that issue there.
%DK also mention that graffiti's issues with CSAM are no worse than whatsapp's


Another important question is whether our implementation is \emph{efficient} enough to support real-world use cases.  While we have not yet undertaken careful performance evaluations, we assert that Graffiti can clearly support a wide range of applications for \emph{sufficiently small} communities, and conversely is unlikely to support certain \emph{massive} applications without connecting to special-purpose servers.  The exact location of the threshold depends on the specifics of the applications.

To justify the lower limit, note that for sufficiently small communities even brute-force broadcast of all posts to all members is quite tractable, so the more efficient distribution provided by our protocols certainly is.   
At the upper end, as discussed with Requirement~\ref{requirements:serverless}, Graffiti will certainly need help to do tasks that involve analyzing massive
portions of the network, like TikTok's For You feed.
%Additionally, total reification also makes tasks
%such as tallying up the millions of likes on
%one of Taylor Swift's posts or compiling all the edits on
%Taylor's Wikipedia page a rather arduous challenge.
%For all of these situations, there must be standards for applications
%to offload massive and repetitive computation to third party services,
%although there must also be ways to freely switch between these
%services so they do not begin to lock the network into a particular design.


Finally, what will the social application
landscape look like when it design is driven by user needs
rather than the profits motives of a few corporate platforms or
the constraints of existing protocols?
Will a new feature as canonical as the like button or the feed emerge?
Or will the ecosystem be marked by its lack of canon?
What would a \emph{tangible} social application feel like?



% as making them becomes more accessible
% and they don't need unanimous support to be

% when not purely directed by broad appeal
% and lock-in potential.
% What social applications are really everything 

% adoption.


% and fostering community self-determination
% can help to create accountability and aid networks that work better
% than paternalistic policing to combat harms like CSAM or 

% In terms of Graffiti's scope, applications like Tik

% The Graffiti API is curre


% Graffiti demonstrates
% % that it is possible to build a wide variety
% % of social interoperating applications on top of a set of primitives.
% % While this paper has described a system that technically works to satisfy our requirements, we have left out some of the social questions: What new potential does this have for harm? Or Good?

% Additionally, while the API is capable of handling many types of applications, how is performance effected aggregating all the likes on one of Taylor Swift's posts, compiling the edits of a popular wikipedia page, or computing TikTok's For You feed. For these, some third party service is necessary, but ideally the user would be able to opt-in or out of these services as freely as with other applications.


% There are questions about its harm. For content like
% CSAM, Graffiti does not make it any more possible
% to host than hosting it on other decentralized systems like the website, or a Mastodon instance,
% or end-to-end encrypted system.
% However, just as how addiction is better handled with
% community care than policing, the healthier communities that Graffiti has the potential to create may provide a pathway
% to get rid of such content without force.
% 
% So much of our lives are online

% \begin{enumerate}
%     \item
%     Recommendation engines and Taylor Swift
%     \item
%     Large media / streaming: 
% A major limitation of the API as it currently stands
% is its support for large media and streaming.
% Currently, in applications like The Gloof, images
% are stored externally and linked in. In other applications
% we did not write about images are encoded with Base64.
% \item 
% Social questions: norms? harms? a world without hierarchical moderation / paternalism? filter bubbles? polarization? CSAM?
% \end{enumerate}

