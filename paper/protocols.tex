\section{System Implementation}
all

To enable interoperability, the login procedure must be portable
so that a user can log in with the
same identity across different applications.
Ideally, login also be portable across different underlying
\emph{protocols} that implement the API.
For example, it should be possible a actor who stores their data on
commodity storage service, like Dropbox, to send a private message
% (by creating an object with an \emph{allowed} list)
to an actor who stores their data on a peer-to-peer network.
Unfortunately, such a universal identity standard does not yet exist
and developing one is outside the scope of this paper.
The Decentralized Identifier (DID) specification~\cite{dids} may serve this purpose one day,
but currently has shortcomings listed in Section~\ref{protocols}.


On the one end
tracker: channl->directories
tracker: channel->objects
There is an in between
tracker: channel->schema
that might be more efficient

Our system, which employs a similar pattern can evolve to shifting threats,
and also different. For example, currently models like end-to-end encryption
allow for trust, but also allow for CSAM. Some users may prefer centralized control.
Additionally, trust is also a shifting target.
Are these systems able to change as cryptographic protocols
are developed and broken?
What happens when the threat model changes? For example,
some might say that centralized services are more trustworthy
than end-to-end-encrypted services because centralized services can
mitigate threats like CSAM.

Graffiti is developed as an API, the user-facing interface to the system.
We have seperated out concerns that directly impact development from
concerns about the underlying system such as scalability,
centralization, end-to-end encryption, etc.
We have built some implementations and sketched out others
that explore different points in the implementation design space.
They show that implementations are possible and also they can coexist,
much as browsers can understand multiple URL schemes, including both HTTP and HTTPS.

The major implementation that we made is the


The Graffiti API is general enough that during the course of our development
we experimented with many different implementations including
a traditional centralized implementation, a peer-to-peer implementation where user
data was stored across users' browser storage, and an implementation that
bootstrapped off existing storage providers like Dropbox and Solid
tied together with a BitTorrent-like tracker service.

While we touch on these other implementations briefly,
the description below is focused primarily on our \emph{canonical} implementation
that we believe, of the options we pursued, strikes
the best balancce between usability, efficiency, and data ownership.
This canonical implementation is also where we focused most of our development effort.
However it is certainly not the only way to implement Graffiti and
in fact one of its strengths is that it is not tied to any particular
implementation.
We imagine a future with coexisting implementations
of Graffiti with bridges between them, much like how the internet protocol
can be built on top of many different physical networks:
copper, optical fiber, radio wi-fi and satellite, etc.

Our canonical implementation of Graffiti consists of four system components:
identity providers, storage pods, a core client library, and a frontend framework.
The identity providers and storage pods are independent servers
--- many of them can exist, and a user can freely choose which ones to use, or even host their own.
A user interact seamlessly with users who use different identity providers and storage pods,
and can migrate their own identity or data between them at any time, preventing lock-in to these services.
The core client library is client-side code that makes
run in parallel while the other
layers build on top of them.

\subsection{Local Implementation}

We have a version of Graffiti that runs locally in the browser or in Node.js.
Any data created in the local version is not accessible other users or even other devices.
While it may seem strange for a social application, it is invaluable for development.
It allows for rapid iteration and testing, and creation of users.
When a user calls "login()" in the local implementation, a popup appears that
simply asks them to type in their actor ID, no additional authentication is required.
Developers can create new data without polluting their own online presence with test data.
And they can do all of this without having spin up a local server.
When they are done, changing their app to a remote version is one line of code.

Since Graffiti implementations can coexist by using different URL schemes,
it is possible to use the local implementation in tandem with other implementations.
This lets a user try out Graffiti applications without having to create an
account. The data they use in the trial will simply only be visible locally but
not available to others. At this time the data cannot be transferred once
a user decides to create an account, but this is scheduled to be added in the future.

Our local implementation builds on PouchDB which provides persistent
storage both in the browser and in node.js.
The local implementation also serves as a reference implementation for the other implementations,
and pieces of its implementation are used in all of the other implementations.

\subsection{Commodity Storage and Tracker}

It is possible to implement Graffiti on top of existing storage providers,
like Dropbox or Solid. Users post objects

Specifically, when a user posts an object, that object is stored in
a large folder of all the user's objects. Then that object is symlinked into
a channel for each folder.

A tracker, like a bittorrent tracker, maps
To avoid a centralized point of control, a user can poll from multiple trackers or
even host their own tracker.

This is useful because existing commodity storage providers are
already widely used, familiar, trusted, robust, reliable.
They have an established economic model with a freemium tier, and paid
service for more storage or bandwidth.
A user does not need to worry about an experimental new service going down.
However they post performance issues:
- A separate network request is required per user and per channel
- It is not possible to do any server side filtering based on the schema of interest
- Dropbox does not provide a portable identity and becomes a centralized point of control.
It is also not possible for access control to happen between platforms like Dropbox and Google Drive.
This is not a problem for Solid, but Solid is unfamiliar and not as robust.
- With Dropbox application developers need to register an application key.

\subsection{Federated Implementation}

The federation implementation is composed of pods that offer
the CRUD operations and discover as first class operations.
It is possible to do server side querying

Identity servers are seperate and we build off of the Solid OIDC protocol.

The "tracker" itself can be implemented on top of the storage servers.
Assuming the network topology will look like email or Mastodon where there are several
big pods (Gmail or Mastodon.social) and a long-tail of small pods.
Users post announcements to the big pods which other users can query
(as a result of being able to query multiple users across the pod).

Implementing the tracker in Graffiti provides all the other benefits of extensibility.
If the tracker is to be updated, it simply needs to change the data,
which can be done easily like adding properties.

The issue of the federated implementation is delegation.
Which pods does a user say they want to do?
It becomes necessary for the Actor to link to a "settings" document
containing delegation and forwarding rules.
This means there are no signatures which would violate the deniability property.

\subsection{Distributed Implementation}

Hypothetically, it is possible to build a distributed implementation of Graffiti too.
In such an implementaiton the data is stored across users' devices.
Caching relays.
The issue is this introduces

[ Insert system diagram here ]

\usepackage{cleveref}
\subsection{Identity Provider}
\label{identity-provider}

To achieve persistent user identities in Graffiti
we use the Solid OpenID Connect specification~\cite{solid-oidc}.
On the surface level, this provides Graffiti applications
with an single-sign-on interface akin to a "Sign In with Google" button
as shown in Figure~\ref{fig:solid-oidc}.
However, while most single-sign-on interfaces only authorize
the client application to interface with a single server,
Solid OIDC authorizes the client application
to communicate with any server on the user's behalf which
is necessary for users to pull data from the multiple storage pods
we describe in~\cref{storage-pods}.
Additionally, the user can choose from any compliant
Solid identity provider including one they host themselves.

\begin{figure}
\label{fig:solid-oidc}
\includegraphics{paper/system/solid-oidc.png}
\caption{A user can choose from several common Solid identity providers
or manualy enter a custom one.}
\end{figure}

The Solid OIDC specification also provides
each user with a webId, a globally unique and human-readable URL
that can be shared like a username or an email address.
At this URL, the user can store a small amount of public data
which we use to implement "delegation".

There are multiple open source implimentations of Solid OIDC servers,
client libraries, and authorization libraries that we use out of the box.
While these implimentations were useful for getting started,
they have some limitations that we would eventually like to address.
None of thse limiations are necessarily at odds with the Solid OIDC
specification, they are either not implemented in the current libraries
or not fully laid out in the specification.

For one, the most popular browser client library is quite large
because it includes other Solid semantic web features not relevant to Graffiti.
This leads to large bundle sizes that can be slow and make Graffiti inaccessable
to users with limited data plans.

Additionally, existing OIDC servers generate their own
webIds, which are often verbose,
\emph{e.g.} \url{janedoe.solidcommunity.net/profile/card#me}.
There is no reason why a user could not come with their own, simple, domain name, for example
\url{id.janedoe.com}, similar to the AT Protocol~\cite{bluesky}. This would also give the users the ability
to switch identity providers while maintaining their identity.
It is not also currently easy to generate throwaway webIds
which could be useful for one-time or anonymous interactions.

Finally, once you are logged in with a webId users are
granted full access rights. Incorporating scoped access rights
into the Solid OIDC specification would allow users to comfortably
interact with new applications without the risk of exposing or losing
their data.

\subsection{Storage Pods}
\label{storage-pods}

Graffiti data is stored and served by storage \emph{pods}.
A person can choose which pod or pods they want to host their data
and at any time migrate to a different pod including one they host themselves.
Users can consume data from any pod.
Altogether, this prevents users from being locked into any one storage provider.

Pods provide two main functionalities:
\begin{enumerate}
\item CRUD style operations for creating, retrieving, updating and deleting data.
\item A \emph{discover} method for querying for data within a pod subject to soft access control (channels)
and hard access control (access control lists).
\end{enumerate}

Almost any online storage provider fulfills the first requirement,
and therefore we have demonstrated that it is possible to implement Graffiti on top both Dropbox and Solid.
However, to implement discovery on top of such services requires a seperate
bittorrent-like \emph{tracker} service and an extremely large number of network requests
to the storage providers that is both expensive and slow.

Therefore, we opted to build our own pods as the primary implementation of Graffiti.
However this suggests that many implementations can coexist and bridges
could be built between them.

\subsubsection{Data Model}

Like ActivityPub, social data is assumed to be comprised of atomic JSON objects.
In addition to their JSON value, they also contain the following metadata
\begin{itemize}
\item \emph{channels} a list of URIs that the object is associated with.
\item \emph{access control list} a list of webIds of users allowed to view the object, or undefined if the object is public.
\item \emph{last modified time} the time the object was last modified which is used for caching.
\item \emph{webId} the webId of the user who created the object.
\item \emph{name} a usually randomly-generated string used to identify the object.
\item \emph{storagePod} the URL of the pod that the object is stored on.
\end{itemize}

\subsubsection{CRUD}

An object is uniquely identified by its storagePod, owner, and name
therefore the URL of an object is of the form \texttt{storagePod/webId/name}.
Users can use PUT, GET, PATCH, and DELETE methods to create, retrieve, update, and delete objects
at these URLs.

PUT, PATCH, and DELETE operations return the object before modifactaion if it exists.

PUT, PATCH, and DELETE operations can only be performed by the owner of an object
- remember that any collaboration or moderation is done by annotating object with new objects.
To authenticate, a user must be logged in with their identity provider.
GET operations do not require authentication unless the object is access controlled.

PATCH operations are implemented with JSON Patch.




One interesting thing is that it shifts the cost of data storage to the poster
rather than the viewer.
For example, if a post goes viral and is viewed by millions of people,
the cost of serving the data is distributed among the millions of viewers
This is unlike traditional social media where the cost of serving the data
is born by the poster.


However for scalability reasons,
we built our own pods that are more specialized for Graffiti.

Graffiti pods offer traditional HTTP methods: PUT, GET, REST,
They also have a `discover' which allows users to find data across the pod.
Finally there are methods

\subsubsection{REST Methods}

Users

At URLS of the form `storagePod/webId/name'.
For example, if the pod is \url{https://pod.graffiti.garden},
the user's webId is \url{https://alice.example.com},
and the object is named \texttt{12345},
the URL would be \url{https://pod.graffiti.garden/https\%3A\%2F\%2Falice.example.com/12345}.

Objects have a value, channels, access control list, and last modified time.
The value is returned in the body,
and the others are returned in headers.
The name and webId are aparent from the URL.

PUT, GET, DELETE, and PATCH are supported.

When putting an object the JSON is in the body.

When getting an object, the object is returned in the body.



When getting an object, channels and access control lists are returned only
if the user is the creator and not for other users.

Patches are done with JSON patch. Seperate patches are passed for value, channels, and acl.

When putting, deleting, or patching and a previous object exists, it is
returned in the response.

\subsubsection{Discover}

\subsubsection{Zero-Knowledge Discovery}

held in storage pods which store and serve data and have minor roles in
limiting access control and filtering content according to queries.

\subsection{Core Client Library}

In addition to interfacing with identity providers,
and storage pods, the core library provides additional functionality
around delegation, announcements, filtering and caching.

The library is written in Typescript and can be used in both the browser and node.js.

\subsubsection{Delegation}

Many decentralized systems use public key cryptography to sign messages,
however this can mean that anything a user writes is undeniable and can
possibly be leaked even if it was meant to be private as we discussed in X.
Instead we use delegation, whereby a user chooses.

Therefore, the client application needs to know which pods a user has delegated
to so that they can know whether to trust the data.

As we saw in \cref{identity-provider}, the Solid OIDC specification
allows a user to store some amount of data at their webId.
That data is a URL which points to a Graffiti object owned by the user
on some pod, and includes global settings.
These settings include a list of

The settings also include forwarding rules which allow a user to

The verification of these settings is done by the client core library.

\subsubsection{Announcements}

When an application makes a \texttt{discover} request to a pod, that request
only returns results from that specific pod. However, users may have posted across
multiple pods. Therefore when a user makes a post in one pod, the client library
also puts "announcements" in several well known pods of the users choosing.
This way, when a user makes a \texttt{discover} request, their client first
makes \texttt{discover} requests for announcements in the users known pods
and then makes \texttt{discover} requests from the pods that have been found.
So long as there is some overlap in the well known pods that users have chosen,
the user will be able to see all of their posts.

These announcements are graffiti objects like any other and so more metadata
or announcement-like objects can be added over time to improve efficiency or
discoverability. For example, it may not be necessary.

\subsubsection{Filtering}

Types help to provide a better software experience.
Anyone can host a pod and so the results are not necessarily trustworthy.
So

\subsubsection{Caching}

A users profile can be

\input{paper/system/front-end}
\subsection{Alternative Implementations}

\subsubsection{Bootstrapping off of Existing Storage Providers}

It is possible to implement Graffiti on top of existing storage providers,
like Dropbox or Solid. When a user posts an object, that object is stored in
a large folder of all the user's objects. Then that object is symlinked into
folders for each channel the user wants to post the object to.

Unlike in our canonical implementation, it is not possible to query for
all objects in a channel by a particular storage provider. Instead, a
separate BitTorrent-like \emph{tracker} service is required to
point users to the appropriate storage provider channel.

Using a commercial storage provider like Dropbox allows for access control but only if users
have a Dropbox account, which locks users into a proprietary service.
Solid does not have this problem. In fact, it is not possible to use the Dropbox API
without being used into a Dropbox account, but it is possible to use the Solid API
without being locked into a Solid account.

The primary reason for not pursuing this implementation is that it requires
an extremely large number of network requests to the storage providers
that is both expensive and slow. Additionally, these storage providers
do not support querying for objects in a particular channel and instead
all the filtering has to be done client-side.

While we expect some users of our canonical implementation to host their own pods,
many will use large pods hosted by others, much like Email or Mastodon.
Only a single network request is required per pod to retrieve all objects in a channel,
whereas a seperate request is required per user and event per object in the Dropbox model.

\subsubsection{Peer-to-Peer Implementation}

While this may have been possible in the PC era, it is not possible in the mobile era.
Most users are not online all the time and can lose their devices or accidentally wipe their data.

