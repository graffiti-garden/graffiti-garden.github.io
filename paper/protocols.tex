\section{System Implementation}
all

Our system, which employs a similar pattern can evolve to shifting threats,
and also different. For example, currently models like end-to-end encryption
allow for trust, but also allow for CSAM. Some users may prefer centralized control.
Additionally, trust is also a shifting target.
Are these systems able to change as cryptographic protocols
are developed and broken?
What happens when the threat model changes? For example,
some might say that centralized services are more trustworthy
than end-to-end-encrypted services because centralized services can
mitigate threats like CSAM.

Graffiti is developed as an API, the user-facing interface to the system.
We have seperated out concerns that directly impact development from
concerns about the underlying system such as scalability,
centralization, end-to-end encryption, etc.
We have built some implementations and sketched out others
that explore different points in the implementation design space.
They show that implementations are possible and also they can coexist,
much as browsers can understand multiple URL schemes, including both HTTP and HTTPS.

The major implementation that we made is the


The Graffiti API is general enough that during the course of our development
we experimented with many different implementations including
a traditional centralized implementation, a peer-to-peer implementation where user
data was stored across users' browser storage, and an implementation that
bootstrapped off existing storage providers like Dropbox and Solid
tied together with a BitTorrent-like tracker service.

While we touch on these other implementations briefly,
the description below is focused primarily on our \emph{canonical} implementation
that we believe, of the options we pursued, strikes
the best balancce between usability, efficiency, and data ownership.
This canonical implementation is also where we focused most of our development effort.
However it is certainly not the only way to implement Graffiti and
in fact one of its strengths is that it is not tied to any particular
implementation.
We imagine a future with coexisting implementations
of Graffiti with bridges between them, much like how the internet protocol
can be built on top of many different physical networks:
copper, optical fiber, radio wi-fi and satellite, etc.

Our canonical implementation of Graffiti consists of four system components:
identity providers, storage pods, a core client library, and a frontend framework.
The identity providers and storage pods are independent servers
--- many of them can exist, and a user can freely choose which ones to use, or even host their own.
A user interact seamlessly with users who use different identity providers and storage pods,
and can migrate their own identity or data between them at any time, preventing lock-in to these services.
The core client library is client-side code that makes
run in parallel while the other
layers build on top of them.

\subsection{Local Implementation}

We have a version of Graffiti that runs locally in the browser or in Node.js.
Any data created in the local version is not accessible other users or even other devices.
While it may seem strange for a social application, it is invaluable for development.
It allows for rapid iteration and testing, and creation of users.
When a user calls "login()" in the local implementation, a popup appears that
simply asks them to type in their actor ID, no additional authentication is required.
Developers can create new data without polluting their own online presence with test data.
And they can do all of this without having spin up a local server.
When they are done, changing their app to a remote version is one line of code.

Since Graffiti implementations can coexist by using different URL schemes,
it is possible to use the local implementation in tandem with other implementations.
This lets a user try out Graffiti applications without having to create an
account. The data they use in the trial will simply only be visible locally but
not available to others. At this time the data cannot be transferred once
a user decides to create an account, but this is scheduled to be added in the future.

Our local implementation builds on PouchDB which provides persistent
storage both in the browser and in node.js.
The local implementation also serves as a reference implementation for the other implementations,
and pieces of its implementation are used in all of the other implementations.

\subsection{Commodity Storage and Tracker}

It is possible to implement Graffiti on top of existing storage providers,
like Dropbox or Solid. Users post objects

Specifically, when a user posts an object, that object is stored in
a large folder of all the user's objects. Then that object is symlinked into
a channel for each folder.

A tracker, like a bittorrent tracker, maps
To avoid a centralized point of control, a user can poll from multiple trackers or
even host their own tracker.

This is useful because existing commodity storage providers are
already widely used, familiar, trusted, robust, reliable.
They have an established economic model with a freemium tier, and paid
service for more storage or bandwidth.
A user does not need to worry about an experimental new service going down.
However they post performance issues:
- A separate network request is required per user and per channel
- It is not possible to do any server side filtering based on the schema of interest
- Dropbox does not provide a portable identity and becomes a centralized point of control.
It is also not possible for access control to happen between platforms like Dropbox and Google Drive.
This is not a problem for Solid, but Solid is unfamiliar and not as robust.
- With Dropbox application developers need to register an application key.

\subsection{Federated Implementation}

The federation implementation is composed of pods that offer
the CRUD operations and discover as first class operations.
It is possible to do server side querying

Identity servers are seperate and we build off of the Solid OIDC protocol.

The "tracker" itself can be implemented on top of the storage servers.
Assuming the network topology will look like email or Mastodon where there are several
big pods (Gmail or Mastodon.social) and a long-tail of small pods.
Users post announcements to the big pods which other users can query
(as a result of being able to query multiple users across the pod).

Implementing the tracker in Graffiti provides all the other benefits of extensibility.
If the tracker is to be updated, it simply needs to change the data,
which can be done easily like adding properties.

The issue of the federated implementation is delegation.
Which pods does a user say they want to do?
It becomes necessary for the Actor to link to a "settings" document
containing delegation and forwarding rules.
This means there are no signatures which would violate the deniability property.

\subsection{Distributed Implementation}

Hypothetically, it is possible to build a distributed implementation of Graffiti too.
In such an implementaiton the data is stored across users' devices.
Caching relays.
The issue is this introduces

[ Insert system diagram here ]

\input{paper/system/identity-provider}
\input{paper/system/storage-pods}
\input{paper/system/core-client}
\input{paper/system/front-end}
\input{paper/system/alternatives}
