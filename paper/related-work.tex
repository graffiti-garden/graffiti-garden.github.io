\section{Related Work}
\label{related-work}

Graffiti, so far as we know, is the only system that makes it
relatively easy to build such a wide variety of interoperating social applications.
Below, we discuss \emph{frameworks} that make it easy to build
\emph{non-\allowbreak{}interoperating} social applications
and \emph{protocols} that support interoperation but
make application development difficult and restrict the design space.
We also discuss other systems with related designs to Graffiti.

\subsection{Frameworks}

Facebook groups, subreddits, and
Slack workspaces are \emph{no-code} frameworks for creating social
applications.
Of course, these sub-applications have limitations,
like a strict user-moderator hierarchy
and predefined feature sets.
However, Reddit and Slack allow moderators to deploy
custom \emph{bots} that enable the
``delta'' ranking system in the subreddit \texttt{r/Change\allowbreak{}My\allowbreak{}View}
~\cite{changemyview}
and the PolicyKit moderation system in Slack~\cite{policykit}.

Déjà Vu is a more expressive framework that allows developers
to declaratively construct a social media application out
of a catalog of primitive \emph{concepts}.
Unfortunately, these concepts are limited in their extensibility.
For example, the ``scoring'' concept cannot be expanded
to support multiple ``reactions''
without creating a new concept and new server code~\cite{dejavu}.

Commercial frameworks, like Google Firebase, allow developers
to build on top of Google's database and identity systems,
enabling pure client-side development.
Firebase plugins for the front-end framework Vue and the low-code tool
Mavo~\cite{mavo} enable developers to write these applications declaratively.
For example, Mavo Chat\footnote{
    \url{https://dmitrysharabin.github.io/mavo-chat/}
} is a real-time chat application written in 156 lines of HTML
with Mavo and Firebase.
%DK could a graffiti remote server be implemented on firebase?

\subsection{Protocols}
\label{related-work:protocols}

Before 2018, social protocols, like Email and IRC,
and later diaspora*~\cite{diaspora} and Secure ScuttleButt~\cite{scuttlebutt},
were designed to support \emph{specific} application types:
messaging for the former and microblogging for the latter.
Since the introduction of ActivityPub~\cite{activitypub},
a new generation of social protocols have
been developed that, technically, support a wider variety of applications.
However, in practice,
building applications on these protocols can be limited by technical
difficulty, barriers to interoperability, or so much interoperability
as to cause context collapse.

All of the protocols we describe below are \emph{federated},
lacking a central point of control or,
at the very least, offering
a ``credible exit'' if one organization in the federation becomes
untrustworthy~\cite{howdecentralizedisbluesky}.

\subsubsection{ActivityPub}
\label{related-work:activitypub}

ActivityPub~\cite{activitypub} is the federated protocol underlying the
``Fediverse,'' which includes the applications
Mastodon (Twitter-like), Pixelfed (Instagram-like), and
Lemmy (Reddit-like).
Graffiti builds on ActivityPub's representation of social artifacts and
activities as extensible objects.
However, some actions in ActivityPub, like moderation, are not
reified into activities, or are
limited in their extensibility by server ``side effects''.

ActivityPub is organized into ``instances'' that conflate
a user's community, moderation service, application,
identity provider, and storage provider into one entity.
Moderation decisions by instance moderators are not reified activities
but actually delete or modify the target content.
This prevents their users from choosing a different moderation
service without leaving their instance and community
behind.
It \emph{also} means that federated content
needs to be moderated by \emph{every} receiving instance,
overwhelming already resource-constrained moderators, causing many
to resort to coarse moderation techniques like ``defederation''
\cite{securingfederatedplatforms, blocklistboundary}.
Note that this redundant labor is not necessary with totally
reified moderation as moderation actions are their own objects
that can be shared across applications.

ActivityPub distributes content using an ``actor model,''
where every actor has an inbox for receiving direct messages, like email,
and an outbox for broadcasting messages, like RSS.
While the actor model is conceptually elegant, it makes it difficult
to implement features as simple as replies.
A replier must send their reply to the original poster's instance which
triggers a server ``side effect'' that forwards the reply to the original
poster's outbox so their followers can discover it.
%DK this means the origial poster can block any replies they don't like from beig received by others
In practice, there is a flow diagram of at least 8 edge cases to consider
to prevent ``ghost replies''~\cite{stateofmastodon}.
Server side effects are needed to facilitate any interaction not centered around identity,
including all of those listed in Table~\ref{concepts:channels-and-applications}.
Additionally, side effects mitigate the autonomous extensibility
(see Requirement~\ref{requirements:autonomous-extensibility})
of ActivityPub
because all involved servers need to implement a side effect before
its corresponding interaction can be reliably used.

Thus there is no ``generic'' ActivityPub server
and custom-built servers
for Mastodon, Lemmy, and PixelFed each consist of hundreds
of thousands of lines of code.
Projects like Hometown, Smalltown, and Glitch\footnote{
    \url{https://glitch.com/fediverse}
} attempt to lower the barrier
to instance creation, but beyond basic configuration panels,
new application designs require modifying the underlying server code~\cite{smalltown,runyourownsocial}.

\subsubsection{Matrix}

The Matrix protocol\footnote{
    \url{https://spec.matrix.org/}
}
provides similar features to IRC and XMPP but content within a ``room''
can be stored across multiple servers and is end-to-end encrypted.
While generally intended for messaging, we mention Matrix because of a
prototype microblogging application built on top of it called \emph{Cerulean}~\cite{cerulean}.

Cerulean creates a ``timeline'' room for each of its users.
Users publish posts to their own timeline room
and follow each other by subscribing to each other's timeline rooms.
When a user posts, their application creates a
public ``thread'' room and replies are copied to both the
thread room and the replier's timeline room.
Matrix's client-server API is flexible enough that this prototype
was written with front-end code over a generic Matrix server.

This is similar to Graffiti's channel-based implementation of microblogging, described
in Section~\ref{case-studies:fig:gloof-and-glitter}.
One difference is that the ``thread'' and ``timeline'' rooms
in Cerulean
must be explicitly created by the poster.
This gives them automatic moderator privileges
under Matrix's rigid ``power level'' moderation
scheme, unlike channels
which are ``permissionless'' but allow arbitrary
moderation structures to be added on top via
reification.
This method also requires that someone is clearly responsible for instantiating the room,
which might not be the case for conversations centered around topics, external media, or locations.

Still, Cerulean's existence indicates that
Matrix might be modified to support the Graffiti API.

\subsubsection{The AT Protocol and Nostr}

The AT Protocol~\cite{bluesky} and Nostr\footnote{
\url{https://nostr.com}
} are two similar federated protocols,
with the AT Protocol being notable for underlying Bluesky.
Both protocols use a combination of ``personal data stores'' and ``relays''
to construct a queryable stream of \emph{all} data in the system.
Within the AT Protocol, this global stream is called the ``firehose''
and intermediate ``app views'' can consume content from this firehose to
construct algorithmic feeds, reducing client-side computation compared
to our own approach, at the cost of standing up custom servers.

A globally queryable stream of data makes contextual separation impossible.
For example, it is impossible for a user, Alice, to post a reply on either protocol
without that reply surfacing in a query for ``anything by Alice.''
This also appears to be causing inter-application collapse, similar to
what we predict in our discussion of Requirement~\ref{requirements:context-differentiation}.
An application called Flashes, built on the AT Protocol,
currently places all images a user has posted to Bluesky within
an Instagram-like grid. Users who want their Flashes feed to be
more curated are simply suggested to use a different account\footnote{
    \url{https://www.youtube.com/live/B7OwcXCE5Rg?t=1655s}
}.

The AT Protocol and Nostr both employ ``stackable
moderation,'' where users can opt-in to various ``labeling'' services.
Labeling-as-moderation is an approachable model that reflects existing expectations of a
top-down moderator-user hierarchy,
hence why we began our discussion of total reification
with \texttt{"Remove"} labels in Section~\ref{concepts:total-reification}.
However, labeling alone does not cover the more general democratic patterns
of moderation that we go on to describe are possible with total reification,
including the reified group membership and document authorship
that we demonstrate in Section~\ref{case-studies}.
Additionally, labelers on the AT Protocol need to run their own labeling servers.

Both protocols include extensible objects, but these objects are not
\emph{autonomously} extensible (see Requirement~\ref{requirements:autonomous-extensibility})
because the relays will not index objects with properties that are unfamiliar to them.
Additionally, both protocols sign all objects with their users' public keys, making
all content \emph{unrepudiable}, violating our Requirement~\ref{requirements:forgiving}.
Finally, the AT Protocol does not currently allow for
any private interactions, including private messaging.

\subsection{Adversarial Interoperability}

Graffiti applications are \emph{adversarially interoperable} with other Graffiti
applications, as per Requirement~\ref{requirements:adversarial-interop}.
However, Graffiti applications do not interoperate with \emph{external}
applications or protocols.
One tool that does is Gobo, which unifies feeds from Reddit, Bluesky,
Mastodon, and, previously, Twitter,
and allows users to customize the algorithms that sort their unified feed~\cite{gobo}.
Matrix also provides ``bridges'' to itself from messaging platforms like
Discord, Slack, and Messenger
\footnote{
    \url{https://matrix.org/ecosystem/bridges/}
}.
It may be possible to build bridges with other services either into or out of
Graffiti as a way to encourage adoption.

\subsection{Channel-Like Models}

Channels, as discussed in Section~\ref{concepts:channels}
are one of the novel aspects about the Graffiti API.
So far, we have discussed channels in relation to the publish-subscribe
pattern~\cite{pubsub}.
Here we mention two other concepts that relate to and shaped our
design of Graffiti's channels.

\subsubsection{Bidirectional Links}

On the World Wide Web, hyperlinks point in one direction, from one website to another.
However, hypertext systems like Xanadu~\cite{xanadu}
and more recently Roam\footnote{\url{https://roamresearch.com}}
and Notion\footnote{\url{https://www.notion.so}}
include \emph{bidirectional} links:
from one document you can see all the other documents that link to it.

Channels can be seen as a way to \emph{selectively} create bidirectional links.
A reply object points \emph{to} the object it is replying to.
Placing the reply in the channel represented by the original post's URL
creates a link in the opposite direction,
pointing \emph{from} the original post object \emph{to} the reply.
The ability for an object's creator to select which of its links should be bidirectional
is critical for managing context collapse.
The general Dexter hypertext model allows for mixed link types, but
as far as we can tell, it was never implemented~\cite{dexter}.

\subsubsection{Object Capabilities}
\label{related-work:ocaps}

%DK "Object capabilities" sounds so much better than "security by obscurity".   Can weintroduce object capabilities  early to describe graffiti's soft access control?
Object capability security is a security model where rights to perform an action
can be transferred to another user by giving the user a reference to that action~\cite{capabilitymyths}.
A channel name can be thought of as a ``read'' capability because knowledge
of a channel name allows an actor to read all public objects in that channel.

The Spritely Institute aims to encapsulate \emph{all}
security aspects of a social application with object capabilities~\cite{spritely}.
Capabilities can indeed be layered to create complex interactions,
including revocation of access.
However, many of these interactions require an agent acting on the user's behalf
that is always \emph{online}, similar to ActivityPub's server side effects.
Since we do not want developers to have to write server code,
we chose to give objects more familiar \texttt{allowed} lists
in addition to channels.

``Write'' capabilities are irrelevant in Graffiti given that total reification enables many coexisting permissions structures
to be built out of only self-writes.
