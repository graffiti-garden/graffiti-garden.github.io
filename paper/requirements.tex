\newtheorem{requirement}{Requirement}

\section{Design Requirements}
\label{requirements}

Graffiti's design is primarily driven by our
first two requirements:
(\ref{requirements:personalization})
the need for communities to personalize
their social applications,
and (\ref{requirements:adversarial-interop})
the need for those applications to interoperate.
Most other requirements work to clarify these two.

Requirements \ref{requirements:autonomous-extensibility}
and \ref{requirements:serverless} expand on how accessible
personalization should be: it should be possible to add new features
without a lengthy standards process, and possible to build
applications without writing server code.
Requirements \ref{requirements:context-differentiation}
and \ref{requirements:forgiving}
address possible antisocial consequences of
interoperation that we need to avoid,
namely by protecting users from ``context collapse,''
and ensuring their ``right to be forgotten.''

Requirement~{\ref{requirements:parallel-implementations}} stands apart to address
how Graffiti might be implemented to uphold these requirements over time, despite
the natural tendency for social technologies to ``enshittify''~{\cite{enshittification}}.

\begin{requirement}[Personalization]
\label{requirements:personalization}
   It should be feasible for communities to build
   their own social media applications.
\end{requirement}

Over the past several years,
works including
\emph{\href{https://publicinfrastructure.org/2023/03/29/the-three-legged-stool/}{The Three-Legged Stool}}~\cite{threeleggedstool},
\href{https://runyourown.social}{runyourown.social}~\cite{runyourownsocial},
and \href{https://pluriverse.world/}{pluriverse.world}\footnote{
\url{https://pluriverse.world}
}
have opined the importance of relatively small,
community-specific social applications.
Existing examples of such applications include the fanfiction
repository \href{https://archiveofourown.org/}{Archive of Our Own},
the hypermedia curation platform \href{https://www.are.na}{are.na},
the movie review site \href{https://letterboxd.com/}{Letterboxd},
and Vermont's local \href{https://frontporchforum.com/}{Front Porch Forum}~\cite{threeleggedstool}.

From their styling, to the features they include (and the ones they omit),
to their moderation policies, to their social norms, these applications
are a reflection of the communities they serve and have a
distinct sense of being \emph{lived-in}
that one-size-fits-all platforms lack.
Most importantly, these applications are designed in close cooperation with the communities themselves,
giving members a sense of empowerment and autonomy over their shared space.
Proponents argue that such online communities are the key to a healthier
digital public sphere and the lessons members learn by participating in such
communities have benefits for civic society more broadly~\cite{threeleggedstool, runyourownsocial, archiveoftheirown}.

The issue is that such applications are difficult to build and maintain,
requiring either significant technical expertise or the money to outsource the work.
Unprivileged communities must resort to other means like creating group chats,
subreddits, or Facebook groups. %, or Mastodon instances.
While these services host many thriving communities,
those communities are often faced with rigid and inadequate feature sets,
limited customization options, and shifting platform designs and policies.
We aim to empower \emph{all} communities with the ability to
build their own applications, giving them autonomy over their digital spaces.

Of course, that is a lofty goal. While the system
we later describe lowers the barrier to participation,
it still requires creating or modifying front-end code, although
we outline steps in Section~\ref{above-and-below:above:future-work}
to lower this barrier further.
This requirement and the ones to follow exist on a spectrum.
We attempt to design for all of them, but are limited by
the scope of this work and the tension that the requirements put
on each other.

\begin{requirement}[Adversarial Interoperability]
\label{requirements:adversarial-interop}
Any application \emph{A} can be built to interoperate with an application \emph{B} without any compliance from application \emph{B}.
\end{requirement}

Communities are not monoliths; they are composed of individuals
with different and evolving preferences.
With interoperability, it becomes possible for individual users and groups of users to explore new application designs
without breaking from their existing communities.
This is especially important when, online, a ``small''
community may consist of tens of thousands of people.

For example, it should be possible for a group of users to
modify an application so that it only shows five posts per day (to combat screen addiction),
without imposing this feature on all other users of the original application.
The original application may eventually include this feature in a configuration panel,
or the applications may drift further apart as their users' needs diverge. However, this
evolution should be driven entirely by users rather than by external pressures
to either conform to a particular design or abandon community altogether.

We use the term ``adversarial interoperability''~\cite{adversarialinterop} because it may not be in the
interest of a large existing application to interoperate with a new application that may co-opt some of its market share.
For example, Facebook, X, and Reddit all have a history of intentionally breaking
the interoperability of third-party applications by changing or restricting access to their APIs.
Even in small group chats, where the power dynamics are less extreme,
an adminstrator-administee hierarchy
can embody an ``implicit feudalism'' that some claim can even
normalize monarchical power structures in society at large~\cite{governablespaces}.
It should never be possible for one entity to hold users
and their data hostage to a particular experience.
Empowerment should not be limited to a community ``as a whole'' but
distributed amongst its subgroups and individual members.

\begin{requirement}[Autonomous Extensibility]
\label{requirements:autonomous-extensibility}
    An application should be able to freely introduce new social features without requiring compliance from existing applications or the system as a whole.
\end{requirement}

%DK it isn't entirely clear how this requirement differs from the "adversarial interop" one or how "expanding to reactions" differs from "limiting to 5 posts per day".
In classic internet forums, users might reply to a post with ``+1'' to signal their support.
In 2009, Facebook created the like button and in 2016 expanded it to a set of five reactions.
Messaging clients, like Signal, now allow users to react to a message with any single emoji and
Slack allows for users to react with multiple emojis, including ones with custom animated icons.

Social applications evolve in unpredictable ways,
so to avoid instant obsolescence, Graffiti must be highly extensible,
allowing for new features
to be developed with relative ease.
Importantly, it should be possible for one application to introduce a new feature
without requiring compliance from other interoperating applications, to avoid
lock applications into a particular set of features,
just as an ecosystem without interoperability can lock users into a
particular set of applications.

For example, if a developer wants to introduce the feature
``reactions to reactions,'' they should not need to draft an RFC,
convince a group of server admins to implement the feature\footnote{
    See Section~\ref{related-work:activitypub}
},
and rewrite the reaction ontology~\cite{ecosystemmoving, herdingdnscamel, semanticwebtwodecades}.
They should be able to build the feature into their application
entirely \emph{on their own}, with other applications able to adopt the
feature as they see fit.

\begin{requirement}[Serverless]
\label{requirements:serverless}
    It should be possible to build a new social application without writing or deploying any server code.
\end{requirement}

Arguably the most difficult part about writing a novel
social application is writing, deploying,
and maintaining server code.
Signal's cofounder wrote
``people don’t want to run their own servers, and never will''~\cite{moxieweb3}.

Of course, Graffiti must involve servers \emph{somewhere} to store data
and pass it around, but we require that all such servers are \emph{generic}
and not tied to any one particular application.
For example, an RSS reader is a ``serverless'' application since
no new server needs to be built specifically for it to function.
The client-side reader simply polls existing RSS servers, each of which outputs data in a generic format.
In Graffiti, both reading \emph{and writing} content must be done serverlessly.

%DK suggested change:
Certain applications have compute or storage demands that make this requirement impossible to meet.
For example, constructing something like TikTok's For You feed requires
analyzing billions of posts, functionality that is not built into our infrastructure and cannot be implemented by client-side applications.
This work focuses on the broad range of applications
that do not require massive machine learning algorithms,
but we conclude with a brief discussion of extending Graffiti to enable them, too.

\begin{requirement}
\label{requirements:context-differentiation}
    Users should be protected from context collapse.
\end{requirement}

Context collapse is a phenomenon that describes the negative consequences of flattening
a person's multiple audiences into a single context.
Context collapse is a well known problem of existing social applications.
For example, if a person's family holds different political beliefs than their friends
but both groups follow them on Facebook, that person may choose to self-suppress their political
speech to avoid angering either side of their collapsed audience.
Either self-suppressing to appeal to the ``lowest-common denominator'' or ignoring the collapse
and simply posting to an ``imagined audience'' can lead to harm for both
the users involved and society at large~\cite{contextcollapse, contextcollapseimpact, contextcollapsequeer, spiralofsilencesocialmedia}.

Users currently leverage the ``siloing'' of existing applications to manage contexts---using
one application to stay in contact with family and another to connect with friends~\cite{whatsappforfamily}.
Interoperability between applications introduces the possibility
of collapsing all of these silos.
For example, both Tinder, a dating app, and Linkedin, a professional networking app, involve the authoring
and browsing of user profiles.
These user profiles have many semantic similarities (name, biography, employment status) and so
it would not be unreasonable to imagine that in a system of interoperable applications,
the profile a user created in one application
naturally shows up in the other.
Of course, this could be confusing or highly embarrassing, and might even get someone fired.

Graffiti must give users affordances to mitigate traditional sources
of context collapse (\emph{e.g.} the ability to limit their posting to suitable subreddit-like scopes)
as well as affordances to mitigate new sources of inter-application context collapse.


\begin{requirement}[Forgiving]
\label{requirements:forgiving}
    Users should be able to delete, edit, and repudiate content they post.
\end{requirement}

Shortly before Elon Musk initiated his acquisition of Twitter, now X,
he created a poll that asked users whether Twitter should add an edit button.
Out of over four million respondents, 73.6\% voted ``yse'' [sic] and 26.4\% ``no.''\footnote{
  X now lets users edit their tweets within 30 minutes of posting but only
  if they pay for X Premium.
} These results are not surprising as many users of Twitter and Facebook regret
their posts for a variety of reasons
including misjudging social norms, posting while extremely emotional or inebriated,
or posting to an unintended audience (cf. context collapse)~\cite{regret, regrettwitter}.

The right to delete one's data was enshrined into law by the European Union's
General Data Protection Regulation as
the \emph{right to erasure}, previously called the \emph{right to be forgotten}~\cite{gdpr}.
We require a stronger claim, that users should also be able to both \emph{edit}
and \emph{repudiate} content that they have posted. Repudiation is the ability to \emph{disown},
as one might do when they say ``that screenshot has been photoshopped,''
or more recently ``that's AI.'' Repudiation is usually possible in centralized platforms
but can be limited by technologies employed in some decentralized platforms,
like public key signatures~\cite{offtherecord}.

The fact that we require mutability (editing and deleting) and repudiation does not rule
out the possibility that users can \emph{opt in} to services that make their content more permanent.
Technologies exist to make a ``forgiving'' system unforgiving but not the other way around.
For example:
\begin{enumerate}
\item
A post could link to content stored on an immutable storage service like IPFS
or a blockchain to prevent erasure.
\item
When a user replies to a post, their application could include the hash
of that post in their reply. Compatible applications would
hide or put a warning over replies to posts whose hashes do not match,
indicating that the post has been edited.
\item
A user could choose to sign their messages with a PGP signature,
as some people do with their emails, to make them non-repudiable.
\end{enumerate}
Some users, like politicians, may face social pressure to use applications
that employ some of these technologies as a form of accountability.
However, to leave the most room for personalization, forgiveness should be the default.

\begin{requirement}[Parallel Implementations]
\label{requirements:parallel-implementations}
    Applications should be able to send and receive data
    from multiple independent implementations of Graffiti.
\end{requirement}

If a single organization were in charge of hosting Graffiti,
history has shown that, even if it started out by making promises of
personalization and interoperability,
over time it would ``enshittify''~{\cite{enshittification}}
and its promises would erode in favor of profit.
This enshittifaction could also extend to other harms,
such as surveillance.
Therefore, we want to ensure that Graffiti can be implemented using
technologies like decentralization and end-to-end encryption
to protect users and their data.

However, just as we do not want to lock users into a particular application (Requirement~{\ref{requirements:adversarial-interop}}),
or developers into a particular set of features (Requirement~{\ref{requirements:autonomous-extensibility}}),
we do not want to lock the system itself into a particular implementation.
The ``ecosystem is moving''~{\cite{ecosystemmoving}}, with both rapidly evolving
technologies and threats, and
leaning too heavily into any one solution would set Graffiti up for failure.

Instead, we require that Graffiti supports multiple implementations
\emph{in parallel}.
This is similar to the World Wide Web, which was designed according to the
``principle of minimal constraint'' to allow for competing implementations,
such as FTP, to coexist with HTTP~{\cite{weavingtheweb}}.
This is made possible by the fact that URLs begin with a \emph{scheme},
such as \texttt{http:}, indicating which implementation should be
used to retrieve the resource.
This allows new implementations, such as HTTPS, to be
incrementally adopted without invalidating existing ones.
