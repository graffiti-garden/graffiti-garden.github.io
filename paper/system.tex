\section{System Implementation}

The Graffiti API is general enough that during the course of our development
we experimented with many different implementations including
a traditional centralized implementation, a peer-to-peer implementation where user
data was stored across users' browser storage, and an implementation that
bootstrapped off existing storage providers like Dropbox and Solid
tied together with a BitTorrent-like tracker service.

While we touch on these other implementations briefly,
the description below is focused primarily on our \emph{canonical} implementation
that we believe, of the options we pursued, strikes
the best balancce between usability, efficiency, and data ownership.
This canonical implementation is also where we focused most of our development effort.
However it is certainly not the only way to implement Graffiti and
in fact one of its strengths is that it is not tied to any particular
implementation.
We imagine a future with coexisting implementations
of Graffiti with bridges between them, much like how the internet protocol
can be built on top of many different physical networks:
copper, optical fiber, radio wi-fi and satellite, etc.

Our canonical implementation of Graffiti consists of four system components:
identity providers, storage pods, a core client library, and a frontend framework.
The identity providers and storage pods are independent servers
--- many of them can exist, and a user can freely choose which ones to use, or even host their own.
A user interact seamlessly with users who use different identity providers and storage pods,
and can migrate their own identity or data between them at any time, preventing lock-in to these services.
The core client library is client-side code that makes
run in parallel while the other
layers build on top of them.

\subsection{Local Implementation}

\subsection{Monolithic Implementation}

\subsection{Commodity Storage and Tracker}

\subsection{Federated Implementation}

\subsection{Distributed Implementation}

[ Insert system diagram here ]

\input{paper/system/identity-provider}
\input{paper/system/storage-pods}
\input{paper/system/core-client}
\input{paper/system/front-end}
\input{paper/system/alternatives}
