\subsection{Storage Pods}
\label{storage-pods}

Graffiti data is stored and served by storage \emph{pods}.
A person can choose which pod or pods they want to host their data
and at any time migrate to a different pod including one they host themselves.
Users can consume data from any pod.
Altogether, this prevents users from being locked into any one storage provider.

Pods provide two main functionalities:
\begin{enumerate}
\item CRUD style operations for creating, retrieving, updating and deleting data.
\item A \emph{discover} method for querying for data within a pod subject to soft access control (channels)
and hard access control (access control lists).
\end{enumerate}

Almost any online storage provider fulfills the first requirement,
and therefore we have demonstrated that it is possible to implement Graffiti on top both Dropbox and Solid.
However, to implement discovery on top of such services requires a seperate
bittorrent-like \emph{tracker} service and an extremely large number of network requests
to the storage providers that is both expensive and slow.

Therefore, we opted to build our own pods as the primary implementation of Graffiti.
However this suggests that many implementations can coexist and bridges
could be built between them.

\subsubsection{Data Model}

Like ActivityPub, social data is assumed to be comprised of atomic JSON objects.
In addition to their JSON value, they also contain the following metadata
\begin{itemize}
\item \emph{channels} a list of URIs that the object is associated with.
\item \emph{access control list} a list of webIds of users allowed to view the object, or undefined if the object is public.
\item \emph{last modified time} the time the object was last modified which is used for caching.
\item \emph{webId} the webId of the user who created the object.
\item \emph{name} a usually randomly-generated string used to identify the object.
\item \emph{storagePod} the URL of the pod that the object is stored on.
\end{itemize}

\subsubsection{CRUD}

An object is uniquely identified by its storagePod, owner, and name
therefore the URL of an object is of the form \texttt{storagePod/webId/name}.
Users can use PUT, GET, PATCH, and DELETE methods to create, retrieve, update, and delete objects
at these URLs.

PUT, PATCH, and DELETE operations return the object before modifactaion if it exists.

PUT, PATCH, and DELETE operations can only be performed by the owner of an object
- remember that any collaboration or moderation is done by annotating object with new objects.
To authenticate, a user must be logged in with their identity provider.
GET operations do not require authentication unless the object is access controlled.

PATCH operations are implemented with JSON Patch.




One interesting thing is that it shifts the cost of data storage to the poster
rather than the viewer.
For example, if a post goes viral and is viewed by millions of people,
the cost of serving the data is distributed among the millions of viewers
This is unlike traditional social media where the cost of serving the data
is born by the poster.


However for scalability reasons,
we built our own pods that are more specialized for Graffiti.

Graffiti pods offer traditional HTTP methods: PUT, GET, REST,
They also have a `discover' which allows users to find data across the pod.
Finally there are methods

\subsubsection{REST Methods}

Users

At URLS of the form `storagePod/webId/name'.
For example, if the pod is \url{https://pod.graffiti.garden},
the user's webId is \url{https://alice.example.com},
and the object is named \texttt{12345},
the URL would be \url{https://pod.graffiti.garden/https\%3A\%2F\%2Falice.example.com/12345}.

Objects have a value, channels, access control list, and last modified time.
The value is returned in the body,
and the others are returned in headers.
The name and webId are aparent from the URL.

PUT, GET, DELETE, and PATCH are supported.

When putting an object the JSON is in the body.

When getting an object, the object is returned in the body.



When getting an object, channels and access control lists are returned only
if the user is the creator and not for other users.

Patches are done with JSON patch. Seperate patches are passed for value, channels, and acl.

When putting, deleting, or patching and a previous object exists, it is
returned in the response.

\subsubsection{Discover}

\subsubsection{Zero-Knowledge Discovery}

held in storage pods which store and serve data and have minor roles in
limiting access control and filtering content according to queries.
